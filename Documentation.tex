\documentclass[10pt, openany, draft]{article}

\usepackage{fancyhdr}
\usepackage{multind}
\usepackage{pstricks}
\usepackage{graphicx}
\usepackage{caption}
\usepackage[yyyymmdd]{datetime}
\renewcommand{\dateseparator}{-}
\usepackage{geometry}
\geometry{letterpaper}
%
% Front Matter
%
\title{Ada Web Server}
\author{Brent Seidel \\ Phoenix, AZ}
\date{ \today }
%========================================================
%%% BEGIN DOCUMENT
\begin{document}
\maketitle
\section{Synopsis}
This Ada web server is a simple server designed to serve a few files and provide an interface to embedded systems.  It is not intended to be a highly flexible high performance server--there are plenty of systems that provide that.  It also does not currently support HTTPS or any authentication methods, so it should not be used in an application that needs security.  However, it does not, by default refer to any external servers so it can be used on an isolated network.

This document describes the configuration of and modifications to the Ada web server.  The amount of customization needed depends on the application.  For some applications, no modifications of the software may be needed.  In other applications, major modifications are needed.

Note: As part of a restructuring, all packages have been prefixed with \texttt{bbs.}  This is part of an effort to allow the code to be used as a library instead of being integrated into the target application.

\section{Installation}
This repository depends on the BBS-Ada repository because of the \texttt{bbs.} prefix.  It is written in Ada 2012 and built using the GNAT Ada tool set.  The following Ada packages are used in various places throughout the system:
\begin{itemize}
  \item \verb|Ada.Characters.Latin_1|
  \item \verb|Ada.Containers.Indefinite_Hashed_Maps|
  \item \verb|Ada.Sequential_IO|
  \item \verb|Ada.Strings|
  \item \verb|Ada.Strings.Equal_Case_Insensitive|
  \item \verb|Ada.Strings.Fixed|
  \item \verb|Ada.Strings.Hash_Case_Insensitive|
  \item \verb|Ada.Strings.Unbounded|
  \item \verb|Ada.Text_IO|
  \item \verb|Ada.Text_IO.Unbounded_IO|
  \item \verb|GNAT.Sockets|
\end{itemize}

\section{Software Structure}
The software consists of a number of packages, each with a spec and a body.  See figure \ref{fig:With}.
\begin{figure}
  \centering
  %LaTeX with PSTricks extensions
%%Creator: inkscape 0.91
%%Please note this file requires PSTricks extensions
\psset{xunit=.5pt,yunit=.5pt,runit=.5pt}
\begin{pspicture}(552.28541817,352.99999507)
{
\newrgbcolor{curcolor}{0 0 0}
\pscustom[linestyle=none,fillstyle=solid,fillcolor=curcolor]
{
\newpath
\moveto(232.38964844,259.91306788)
\lineto(234.20019531,252.49509913)
\lineto(236.03710938,259.91306788)
\lineto(237.8125,259.91306788)
\lineto(239.65820312,252.53904444)
\lineto(241.58300781,259.91306788)
\lineto(243.16503906,259.91306788)
\lineto(240.43164062,250.49998194)
\lineto(238.78808594,250.49998194)
\lineto(236.87207031,257.78611475)
\lineto(235.01757812,250.49998194)
\lineto(233.37402344,250.49998194)
\lineto(230.65820312,259.91306788)
\lineto(232.38964844,259.91306788)
\closepath
}
}
{
\newrgbcolor{curcolor}{0 0 0}
\pscustom[linestyle=none,fillstyle=solid,fillcolor=curcolor]
{
\newpath
\moveto(248.58789062,260.12400538)
\curveto(249.25585938,260.12400538)(249.90332031,259.96580225)(250.53027344,259.649396)
\curveto(251.15722656,259.33884913)(251.63476562,258.93455225)(251.96289062,258.43650538)
\curveto(252.27929688,257.961896)(252.49023438,257.40818506)(252.59570312,256.77537256)
\curveto(252.68945312,256.34177881)(252.73632812,255.65037256)(252.73632812,254.70115381)
\lineto(245.83691406,254.70115381)
\curveto(245.86621094,253.74607569)(246.09179688,252.97849756)(246.51367188,252.39841944)
\curveto(246.93554688,251.82420069)(247.58886719,251.53709131)(248.47363281,251.53709131)
\curveto(249.29980469,251.53709131)(249.95898438,251.80955225)(250.45117188,252.35447413)
\curveto(250.73242188,252.67088038)(250.93164062,253.03709131)(251.04882812,253.45310694)
\lineto(252.60449219,253.45310694)
\curveto(252.56347656,253.10740381)(252.42578125,252.72068506)(252.19140625,252.29295069)
\curveto(251.96289062,251.87107569)(251.70507812,251.52537256)(251.41796875,251.25584131)
\curveto(250.9375,250.78709131)(250.34277344,250.47068506)(249.63378906,250.30662256)
\curveto(249.25292969,250.21287256)(248.82226562,250.16599756)(248.34179688,250.16599756)
\curveto(247.16992188,250.16599756)(246.17675781,250.59080225)(245.36230469,251.44041163)
\curveto(244.54785156,252.29588038)(244.140625,253.49119288)(244.140625,255.02634913)
\curveto(244.140625,256.53806788)(244.55078125,257.76560694)(245.37109375,258.70896631)
\curveto(246.19140625,259.65232569)(247.26367188,260.12400538)(248.58789062,260.12400538)
\closepath
\moveto(251.11035156,255.95798975)
\curveto(251.04589844,256.64353663)(250.89648438,257.19138819)(250.66210938,257.60154444)
\curveto(250.22851562,258.36326319)(249.50488281,258.74412256)(248.49121094,258.74412256)
\curveto(247.76464844,258.74412256)(247.15527344,258.48045069)(246.66308594,257.95310694)
\curveto(246.17089844,257.43162256)(245.91015625,256.7665835)(245.88085938,255.95798975)
\lineto(251.11035156,255.95798975)
\closepath
\moveto(248.43847656,260.1415835)
\lineto(248.43847656,260.1415835)
\closepath
}
}
{
\newrgbcolor{curcolor}{0 0 0}
\pscustom[linestyle=none,fillstyle=solid,fillcolor=curcolor]
{
\newpath
\moveto(254.56445312,263.45506006)
\lineto(256.10253906,263.45506006)
\lineto(256.10253906,258.77048975)
\curveto(256.44824219,259.22166163)(256.86132812,259.56443506)(257.34179688,259.79881006)
\curveto(257.82226562,260.03904444)(258.34375,260.15916163)(258.90625,260.15916163)
\curveto(260.078125,260.15916163)(261.02734375,259.75486475)(261.75390625,258.946271)
\curveto(262.48632812,258.14353663)(262.85253906,256.95701319)(262.85253906,255.38670069)
\curveto(262.85253906,253.89841944)(262.4921875,252.66209131)(261.77148438,251.67771631)
\curveto(261.05078125,250.69334131)(260.05175781,250.20115381)(258.77441406,250.20115381)
\curveto(258.05957031,250.20115381)(257.45605469,250.37400538)(256.96386719,250.7197085)
\curveto(256.67089844,250.92478663)(256.35742188,251.25291163)(256.0234375,251.7040835)
\lineto(256.0234375,250.49998194)
\lineto(254.56445312,250.49998194)
\lineto(254.56445312,263.45506006)
\closepath
\moveto(258.67773438,251.59861475)
\curveto(259.53320312,251.59861475)(260.171875,251.9384585)(260.59375,252.618146)
\curveto(261.02148438,253.2978335)(261.23535156,254.19431788)(261.23535156,255.30759913)
\curveto(261.23535156,256.2978335)(261.02148438,257.118146)(260.59375,257.76853663)
\curveto(260.171875,258.41892725)(259.54785156,258.74412256)(258.72167969,258.74412256)
\curveto(258.00097656,258.74412256)(257.36816406,258.477521)(256.82324219,257.94431788)
\curveto(256.28417969,257.41111475)(256.01464844,256.5322085)(256.01464844,255.30759913)
\curveto(256.01464844,254.4228335)(256.12597656,253.70506006)(256.34863281,253.15427881)
\curveto(256.76464844,252.11716944)(257.54101562,251.59861475)(258.67773438,251.59861475)
\closepath
}
}
{
\newrgbcolor{curcolor}{0 0 0}
\pscustom[linestyle=none,fillstyle=solid,fillcolor=curcolor]
{
\newpath
\moveto(263.546875,248.24998194)
\lineto(263.546875,249.13767725)
\lineto(273.55761719,249.13767725)
\lineto(273.55761719,248.24998194)
\lineto(263.546875,248.24998194)
\closepath
}
}
{
\newrgbcolor{curcolor}{0 0 0}
\pscustom[linestyle=none,fillstyle=solid,fillcolor=curcolor]
{
\newpath
\moveto(275.66699219,253.45310694)
\curveto(275.71386719,252.92576319)(275.84570312,252.52146631)(276.0625,252.24021631)
\curveto(276.4609375,251.73045069)(277.15234375,251.47556788)(278.13671875,251.47556788)
\curveto(278.72265625,251.47556788)(279.23828125,251.60154444)(279.68359375,251.85349756)
\curveto(280.12890625,252.11131006)(280.3515625,252.50681788)(280.3515625,253.040021)
\curveto(280.3515625,253.44431788)(280.17285156,253.75193506)(279.81542969,253.96287256)
\curveto(279.58691406,254.09177881)(279.13574219,254.24119288)(278.46191406,254.41111475)
\lineto(277.20507812,254.727521)
\curveto(276.40234375,254.92673975)(275.81054688,255.149396)(275.4296875,255.39548975)
\curveto(274.75,255.82322413)(274.41015625,256.415021)(274.41015625,257.17088038)
\curveto(274.41015625,258.06150538)(274.72949219,258.7822085)(275.36816406,259.33298975)
\curveto(276.01269531,259.883771)(276.87695312,260.15916163)(277.9609375,260.15916163)
\curveto(279.37890625,260.15916163)(280.40136719,259.743146)(281.02832031,258.91111475)
\curveto(281.42089844,258.383771)(281.61132812,257.81541163)(281.59960938,257.20603663)
\lineto(280.10546875,257.20603663)
\curveto(280.07617188,257.5634585)(279.95019531,257.88865381)(279.72753906,258.18162256)
\curveto(279.36425781,258.59763819)(278.734375,258.805646)(277.83789062,258.805646)
\curveto(277.24023438,258.805646)(276.78613281,258.69138819)(276.47558594,258.46287256)
\curveto(276.17089844,258.23435694)(276.01855469,257.93259913)(276.01855469,257.55759913)
\curveto(276.01855469,257.14744288)(276.22070312,256.81931788)(276.625,256.57322413)
\curveto(276.859375,256.42673975)(277.20507812,256.2978335)(277.66210938,256.18650538)
\lineto(278.70800781,255.93162256)
\curveto(279.84472656,255.65623194)(280.60644531,255.38963038)(280.99316406,255.13181788)
\curveto(281.60839844,254.727521)(281.91601562,254.09177881)(281.91601562,253.22459131)
\curveto(281.91601562,252.38670069)(281.59667969,251.66306788)(280.95800781,251.05369288)
\curveto(280.32519531,250.44431788)(279.35839844,250.13963038)(278.05761719,250.13963038)
\curveto(276.65722656,250.13963038)(275.6640625,250.45603663)(275.078125,251.08884913)
\curveto(274.49804688,251.727521)(274.1875,252.51560694)(274.14648438,253.45310694)
\lineto(275.66699219,253.45310694)
\closepath
\moveto(278.00488281,260.1415835)
\lineto(278.00488281,260.1415835)
\closepath
}
}
{
\newrgbcolor{curcolor}{0 0 0}
\pscustom[linestyle=none,fillstyle=solid,fillcolor=curcolor]
{
\newpath
\moveto(287.68164062,260.12400538)
\curveto(288.34960938,260.12400538)(288.99707031,259.96580225)(289.62402344,259.649396)
\curveto(290.25097656,259.33884913)(290.72851562,258.93455225)(291.05664062,258.43650538)
\curveto(291.37304688,257.961896)(291.58398438,257.40818506)(291.68945312,256.77537256)
\curveto(291.78320312,256.34177881)(291.83007812,255.65037256)(291.83007812,254.70115381)
\lineto(284.93066406,254.70115381)
\curveto(284.95996094,253.74607569)(285.18554688,252.97849756)(285.60742188,252.39841944)
\curveto(286.02929688,251.82420069)(286.68261719,251.53709131)(287.56738281,251.53709131)
\curveto(288.39355469,251.53709131)(289.05273438,251.80955225)(289.54492188,252.35447413)
\curveto(289.82617188,252.67088038)(290.02539062,253.03709131)(290.14257812,253.45310694)
\lineto(291.69824219,253.45310694)
\curveto(291.65722656,253.10740381)(291.51953125,252.72068506)(291.28515625,252.29295069)
\curveto(291.05664062,251.87107569)(290.79882812,251.52537256)(290.51171875,251.25584131)
\curveto(290.03125,250.78709131)(289.43652344,250.47068506)(288.72753906,250.30662256)
\curveto(288.34667969,250.21287256)(287.91601562,250.16599756)(287.43554688,250.16599756)
\curveto(286.26367188,250.16599756)(285.27050781,250.59080225)(284.45605469,251.44041163)
\curveto(283.64160156,252.29588038)(283.234375,253.49119288)(283.234375,255.02634913)
\curveto(283.234375,256.53806788)(283.64453125,257.76560694)(284.46484375,258.70896631)
\curveto(285.28515625,259.65232569)(286.35742188,260.12400538)(287.68164062,260.12400538)
\closepath
\moveto(290.20410156,255.95798975)
\curveto(290.13964844,256.64353663)(289.99023438,257.19138819)(289.75585938,257.60154444)
\curveto(289.32226562,258.36326319)(288.59863281,258.74412256)(287.58496094,258.74412256)
\curveto(286.85839844,258.74412256)(286.24902344,258.48045069)(285.75683594,257.95310694)
\curveto(285.26464844,257.43162256)(285.00390625,256.7665835)(284.97460938,255.95798975)
\lineto(290.20410156,255.95798975)
\closepath
\moveto(287.53222656,260.1415835)
\lineto(287.53222656,260.1415835)
\closepath
}
}
{
\newrgbcolor{curcolor}{0 0 0}
\pscustom[linestyle=none,fillstyle=solid,fillcolor=curcolor]
{
\newpath
\moveto(293.82519531,259.91306788)
\lineto(295.328125,259.91306788)
\lineto(295.328125,258.28709131)
\curveto(295.45117188,258.60349756)(295.75292969,258.98728663)(296.23339844,259.4384585)
\curveto(296.71386719,259.89548975)(297.26757812,260.12400538)(297.89453125,260.12400538)
\curveto(297.92382812,260.12400538)(297.97363281,260.12107569)(298.04394531,260.11521631)
\curveto(298.11425781,260.10935694)(298.234375,260.09763819)(298.40429688,260.08006006)
\lineto(298.40429688,258.41013819)
\curveto(298.31054688,258.42771631)(298.22265625,258.43943506)(298.140625,258.44529444)
\curveto(298.06445312,258.45115381)(297.97949219,258.4540835)(297.88574219,258.4540835)
\curveto(297.08886719,258.4540835)(296.4765625,258.196271)(296.04882812,257.680646)
\curveto(295.62109375,257.17088038)(295.40722656,256.58201319)(295.40722656,255.91404444)
\lineto(295.40722656,250.49998194)
\lineto(293.82519531,250.49998194)
\lineto(293.82519531,259.91306788)
\closepath
}
}
{
\newrgbcolor{curcolor}{0 0 0}
\pscustom[linestyle=none,fillstyle=solid,fillcolor=curcolor]
{
\newpath
\moveto(300.56640625,259.91306788)
\lineto(303.08007812,252.24900538)
\lineto(305.70800781,259.91306788)
\lineto(307.43945312,259.91306788)
\lineto(303.88867188,250.49998194)
\lineto(302.20117188,250.49998194)
\lineto(298.72949219,259.91306788)
\lineto(300.56640625,259.91306788)
\closepath
}
}
{
\newrgbcolor{curcolor}{0 0 0}
\pscustom[linestyle=none,fillstyle=solid,fillcolor=curcolor]
{
\newpath
\moveto(312.71289062,260.12400538)
\curveto(313.38085938,260.12400538)(314.02832031,259.96580225)(314.65527344,259.649396)
\curveto(315.28222656,259.33884913)(315.75976562,258.93455225)(316.08789062,258.43650538)
\curveto(316.40429688,257.961896)(316.61523438,257.40818506)(316.72070312,256.77537256)
\curveto(316.81445312,256.34177881)(316.86132812,255.65037256)(316.86132812,254.70115381)
\lineto(309.96191406,254.70115381)
\curveto(309.99121094,253.74607569)(310.21679688,252.97849756)(310.63867188,252.39841944)
\curveto(311.06054688,251.82420069)(311.71386719,251.53709131)(312.59863281,251.53709131)
\curveto(313.42480469,251.53709131)(314.08398438,251.80955225)(314.57617188,252.35447413)
\curveto(314.85742188,252.67088038)(315.05664062,253.03709131)(315.17382812,253.45310694)
\lineto(316.72949219,253.45310694)
\curveto(316.68847656,253.10740381)(316.55078125,252.72068506)(316.31640625,252.29295069)
\curveto(316.08789062,251.87107569)(315.83007812,251.52537256)(315.54296875,251.25584131)
\curveto(315.0625,250.78709131)(314.46777344,250.47068506)(313.75878906,250.30662256)
\curveto(313.37792969,250.21287256)(312.94726562,250.16599756)(312.46679688,250.16599756)
\curveto(311.29492188,250.16599756)(310.30175781,250.59080225)(309.48730469,251.44041163)
\curveto(308.67285156,252.29588038)(308.265625,253.49119288)(308.265625,255.02634913)
\curveto(308.265625,256.53806788)(308.67578125,257.76560694)(309.49609375,258.70896631)
\curveto(310.31640625,259.65232569)(311.38867188,260.12400538)(312.71289062,260.12400538)
\closepath
\moveto(315.23535156,255.95798975)
\curveto(315.17089844,256.64353663)(315.02148438,257.19138819)(314.78710938,257.60154444)
\curveto(314.35351562,258.36326319)(313.62988281,258.74412256)(312.61621094,258.74412256)
\curveto(311.88964844,258.74412256)(311.28027344,258.48045069)(310.78808594,257.95310694)
\curveto(310.29589844,257.43162256)(310.03515625,256.7665835)(310.00585938,255.95798975)
\lineto(315.23535156,255.95798975)
\closepath
\moveto(312.56347656,260.1415835)
\lineto(312.56347656,260.1415835)
\closepath
}
}
{
\newrgbcolor{curcolor}{0 0 0}
\pscustom[linestyle=none,fillstyle=solid,fillcolor=curcolor]
{
\newpath
\moveto(318.85644531,259.91306788)
\lineto(320.359375,259.91306788)
\lineto(320.359375,258.28709131)
\curveto(320.48242188,258.60349756)(320.78417969,258.98728663)(321.26464844,259.4384585)
\curveto(321.74511719,259.89548975)(322.29882812,260.12400538)(322.92578125,260.12400538)
\curveto(322.95507812,260.12400538)(323.00488281,260.12107569)(323.07519531,260.11521631)
\curveto(323.14550781,260.10935694)(323.265625,260.09763819)(323.43554688,260.08006006)
\lineto(323.43554688,258.41013819)
\curveto(323.34179688,258.42771631)(323.25390625,258.43943506)(323.171875,258.44529444)
\curveto(323.09570312,258.45115381)(323.01074219,258.4540835)(322.91699219,258.4540835)
\curveto(322.12011719,258.4540835)(321.5078125,258.196271)(321.08007812,257.680646)
\curveto(320.65234375,257.17088038)(320.43847656,256.58201319)(320.43847656,255.91404444)
\lineto(320.43847656,250.49998194)
\lineto(318.85644531,250.49998194)
\lineto(318.85644531,259.91306788)
\closepath
}
}
{
\newrgbcolor{curcolor}{0 0 0}
\pscustom[linewidth=1,linecolor=curcolor]
{
\newpath
\moveto(220.5,270.4999972)
\lineto(340.5,270.4999972)
\lineto(340.5,240.4999972)
\lineto(220.5,240.4999972)
\closepath
}
}
{
\newrgbcolor{curcolor}{0 0 0}
\pscustom[linestyle=none,fillstyle=solid,fillcolor=curcolor]
{
\newpath
\moveto(261.82714844,345.41113001)
\lineto(264.33203125,345.41113001)
\lineto(268.04101562,334.49511439)
\lineto(271.72363281,345.41113001)
\lineto(274.20214844,345.41113001)
\lineto(274.20214844,332.4999972)
\lineto(272.54101562,332.4999972)
\lineto(272.54101562,340.12011439)
\curveto(272.54101562,340.38378626)(272.546875,340.8203097)(272.55859375,341.4296847)
\curveto(272.5703125,342.0390597)(272.57617188,342.69238001)(272.57617188,343.38964564)
\lineto(268.89355469,332.4999972)
\lineto(267.16210938,332.4999972)
\lineto(263.453125,343.38964564)
\lineto(263.453125,342.99413782)
\curveto(263.453125,342.67773157)(263.45898438,342.19433314)(263.47070312,341.54394251)
\curveto(263.48828125,340.89941126)(263.49707031,340.42480189)(263.49707031,340.12011439)
\lineto(263.49707031,332.4999972)
\lineto(261.82714844,332.4999972)
\lineto(261.82714844,345.41113001)
\closepath
}
}
{
\newrgbcolor{curcolor}{0 0 0}
\pscustom[linestyle=none,fillstyle=solid,fillcolor=curcolor]
{
\newpath
\moveto(277.88476562,335.00488001)
\curveto(277.88476562,334.54784876)(278.05175781,334.1874972)(278.38574219,333.92382532)
\curveto(278.71972656,333.66015345)(279.11523438,333.52831751)(279.57226562,333.52831751)
\curveto(280.12890625,333.52831751)(280.66796875,333.65722376)(281.18945312,333.91503626)
\curveto(282.06835938,334.34277064)(282.5078125,335.04296595)(282.5078125,336.0156222)
\lineto(282.5078125,337.29003626)
\curveto(282.31445312,337.16698939)(282.06542969,337.06445032)(281.76074219,336.98241907)
\curveto(281.45605469,336.90038782)(281.15722656,336.84179407)(280.86425781,336.80663782)
\lineto(279.90625,336.68359095)
\curveto(279.33203125,336.60741907)(278.90136719,336.48730189)(278.61425781,336.32323939)
\curveto(278.12792969,336.04784876)(277.88476562,335.60839564)(277.88476562,335.00488001)
\closepath
\moveto(281.71679688,338.20409876)
\curveto(282.08007812,338.25097376)(282.32324219,338.40331751)(282.44628906,338.66113001)
\curveto(282.51660156,338.80175501)(282.55175781,339.00390345)(282.55175781,339.26757532)
\curveto(282.55175781,339.80663782)(282.35839844,340.19628626)(281.97167969,340.43652064)
\curveto(281.59082031,340.68261439)(281.04296875,340.80566126)(280.328125,340.80566126)
\curveto(279.50195312,340.80566126)(278.91601562,340.58300501)(278.5703125,340.13769251)
\curveto(278.37695312,339.89159876)(278.25097656,339.52538782)(278.19238281,339.0390597)
\lineto(276.71582031,339.0390597)
\curveto(276.74511719,340.19921595)(277.12011719,341.00488001)(277.84082031,341.45605189)
\curveto(278.56738281,341.91308314)(279.40820312,342.14159876)(280.36328125,342.14159876)
\curveto(281.47070312,342.14159876)(282.37011719,341.93066126)(283.06152344,341.50878626)
\curveto(283.74707031,341.08691126)(284.08984375,340.43066126)(284.08984375,339.54003626)
\lineto(284.08984375,334.1171847)
\curveto(284.08984375,333.9531222)(284.12207031,333.82128626)(284.18652344,333.72167689)
\curveto(284.25683594,333.62206751)(284.40039062,333.57226282)(284.6171875,333.57226282)
\curveto(284.6875,333.57226282)(284.76660156,333.57519251)(284.85449219,333.58105189)
\curveto(284.94238281,333.59277064)(285.03613281,333.60741907)(285.13574219,333.6249972)
\lineto(285.13574219,332.45605189)
\curveto(284.88964844,332.38573939)(284.70214844,332.34179407)(284.57324219,332.32421595)
\curveto(284.44433594,332.30663782)(284.26855469,332.29784876)(284.04589844,332.29784876)
\curveto(283.50097656,332.29784876)(283.10546875,332.49120814)(282.859375,332.87792689)
\curveto(282.73046875,333.08300501)(282.63964844,333.37304407)(282.58691406,333.74804407)
\curveto(282.26464844,333.32616907)(281.80175781,332.95995814)(281.19824219,332.64941126)
\curveto(280.59472656,332.33886439)(279.9296875,332.18359095)(279.203125,332.18359095)
\curveto(278.33007812,332.18359095)(277.61523438,332.44726282)(277.05859375,332.97460657)
\curveto(276.5078125,333.5078097)(276.23242188,334.17284876)(276.23242188,334.96972376)
\curveto(276.23242188,335.84277064)(276.50488281,336.51952845)(277.04980469,336.9999972)
\curveto(277.59472656,337.48046595)(278.30957031,337.77636439)(279.19433594,337.88769251)
\lineto(281.71679688,338.20409876)
\closepath
\moveto(280.40722656,342.14159876)
\lineto(280.40722656,342.14159876)
\closepath
}
}
{
\newrgbcolor{curcolor}{0 0 0}
\pscustom[linestyle=none,fillstyle=solid,fillcolor=curcolor]
{
\newpath
\moveto(286.69140625,341.86913782)
\lineto(288.29980469,341.86913782)
\lineto(288.29980469,332.4999972)
\lineto(286.69140625,332.4999972)
\lineto(286.69140625,341.86913782)
\closepath
\moveto(286.69140625,345.41113001)
\lineto(288.29980469,345.41113001)
\lineto(288.29980469,343.61816126)
\lineto(286.69140625,343.61816126)
\lineto(286.69140625,345.41113001)
\closepath
}
}
{
\newrgbcolor{curcolor}{0 0 0}
\pscustom[linestyle=none,fillstyle=solid,fillcolor=curcolor]
{
\newpath
\moveto(290.69921875,341.91308314)
\lineto(292.20214844,341.91308314)
\lineto(292.20214844,340.57714564)
\curveto(292.64746094,341.12792689)(293.11914062,341.5234347)(293.6171875,341.76366907)
\curveto(294.11523438,342.00390345)(294.66894531,342.12402064)(295.27832031,342.12402064)
\curveto(296.61425781,342.12402064)(297.51660156,341.65820032)(297.98535156,340.7265597)
\curveto(298.24316406,340.21679407)(298.37207031,339.48730189)(298.37207031,338.53808314)
\lineto(298.37207031,332.4999972)
\lineto(296.76367188,332.4999972)
\lineto(296.76367188,338.43261439)
\curveto(296.76367188,339.00683314)(296.67871094,339.46972376)(296.50878906,339.82128626)
\curveto(296.22753906,340.40722376)(295.71777344,340.70019251)(294.97949219,340.70019251)
\curveto(294.60449219,340.70019251)(294.296875,340.66210657)(294.05664062,340.5859347)
\curveto(293.62304688,340.45702845)(293.2421875,340.19921595)(292.9140625,339.8124972)
\curveto(292.65039062,339.50195032)(292.47753906,339.1796847)(292.39550781,338.84570032)
\curveto(292.31933594,338.51757532)(292.28125,338.04589564)(292.28125,337.43066126)
\lineto(292.28125,332.4999972)
\lineto(290.69921875,332.4999972)
\lineto(290.69921875,341.91308314)
\closepath
\moveto(294.41699219,342.14159876)
\lineto(294.41699219,342.14159876)
\closepath
}
}
{
\newrgbcolor{curcolor}{0 0 0}
\pscustom[linewidth=1,linecolor=curcolor]
{
\newpath
\moveto(220.5,352.4999972)
\lineto(340.5,352.4999972)
\lineto(340.5,322.4999972)
\lineto(220.5,322.4999972)
\closepath
}
}
{
\newrgbcolor{curcolor}{0 0 0}
\pscustom[linewidth=1,linecolor=curcolor]
{
\newpath
\moveto(0.5,192.50001246)
\lineto(120.5,192.50001246)
\lineto(120.5,162.50001246)
\lineto(0.5,162.50001246)
\closepath
}
}
{
\newrgbcolor{curcolor}{0 0 0}
\pscustom[linestyle=none,fillstyle=solid,fillcolor=curcolor]
{
\newpath
\moveto(11.53710938,185.45506006)
\lineto(13.07519531,185.45506006)
\lineto(13.07519531,180.77048975)
\curveto(13.42089844,181.22166163)(13.83398438,181.56443506)(14.31445312,181.79881006)
\curveto(14.79492188,182.03904444)(15.31640625,182.15916163)(15.87890625,182.15916163)
\curveto(17.05078125,182.15916163)(18,181.75486475)(18.7265625,180.946271)
\curveto(19.45898438,180.14353663)(19.82519531,178.95701319)(19.82519531,177.38670069)
\curveto(19.82519531,175.89841944)(19.46484375,174.66209131)(18.74414062,173.67771631)
\curveto(18.0234375,172.69334131)(17.02441406,172.20115381)(15.74707031,172.20115381)
\curveto(15.03222656,172.20115381)(14.42871094,172.37400538)(13.93652344,172.7197085)
\curveto(13.64355469,172.92478663)(13.33007812,173.25291163)(12.99609375,173.7040835)
\lineto(12.99609375,172.49998194)
\lineto(11.53710938,172.49998194)
\lineto(11.53710938,185.45506006)
\closepath
\moveto(15.65039062,173.59861475)
\curveto(16.50585938,173.59861475)(17.14453125,173.9384585)(17.56640625,174.618146)
\curveto(17.99414062,175.2978335)(18.20800781,176.19431788)(18.20800781,177.30759913)
\curveto(18.20800781,178.2978335)(17.99414062,179.118146)(17.56640625,179.76853663)
\curveto(17.14453125,180.41892725)(16.52050781,180.74412256)(15.69433594,180.74412256)
\curveto(14.97363281,180.74412256)(14.34082031,180.477521)(13.79589844,179.94431788)
\curveto(13.25683594,179.41111475)(12.98730469,178.5322085)(12.98730469,177.30759913)
\curveto(12.98730469,176.4228335)(13.09863281,175.70506006)(13.32128906,175.15427881)
\curveto(13.73730469,174.11716944)(14.51367188,173.59861475)(15.65039062,173.59861475)
\closepath
}
}
{
\newrgbcolor{curcolor}{0 0 0}
\pscustom[linestyle=none,fillstyle=solid,fillcolor=curcolor]
{
\newpath
\moveto(21.6796875,181.86912256)
\lineto(23.28808594,181.86912256)
\lineto(23.28808594,172.49998194)
\lineto(21.6796875,172.49998194)
\lineto(21.6796875,181.86912256)
\closepath
\moveto(21.6796875,185.41111475)
\lineto(23.28808594,185.41111475)
\lineto(23.28808594,183.618146)
\lineto(21.6796875,183.618146)
\lineto(21.6796875,185.41111475)
\closepath
}
}
{
\newrgbcolor{curcolor}{0 0 0}
\pscustom[linestyle=none,fillstyle=solid,fillcolor=curcolor]
{
\newpath
\moveto(25.6875,181.91306788)
\lineto(27.19042969,181.91306788)
\lineto(27.19042969,180.57713038)
\curveto(27.63574219,181.12791163)(28.10742188,181.52341944)(28.60546875,181.76365381)
\curveto(29.10351562,182.00388819)(29.65722656,182.12400538)(30.26660156,182.12400538)
\curveto(31.60253906,182.12400538)(32.50488281,181.65818506)(32.97363281,180.72654444)
\curveto(33.23144531,180.21677881)(33.36035156,179.48728663)(33.36035156,178.53806788)
\lineto(33.36035156,172.49998194)
\lineto(31.75195312,172.49998194)
\lineto(31.75195312,178.43259913)
\curveto(31.75195312,179.00681788)(31.66699219,179.4697085)(31.49707031,179.821271)
\curveto(31.21582031,180.4072085)(30.70605469,180.70017725)(29.96777344,180.70017725)
\curveto(29.59277344,180.70017725)(29.28515625,180.66209131)(29.04492188,180.58591944)
\curveto(28.61132812,180.45701319)(28.23046875,180.19920069)(27.90234375,179.81248194)
\curveto(27.63867188,179.50193506)(27.46582031,179.17966944)(27.38378906,178.84568506)
\curveto(27.30761719,178.51756006)(27.26953125,178.04588038)(27.26953125,177.430646)
\lineto(27.26953125,172.49998194)
\lineto(25.6875,172.49998194)
\lineto(25.6875,181.91306788)
\closepath
\moveto(29.40527344,182.1415835)
\lineto(29.40527344,182.1415835)
\closepath
}
}
{
\newrgbcolor{curcolor}{0 0 0}
\pscustom[linestyle=none,fillstyle=solid,fillcolor=curcolor]
{
\newpath
\moveto(36.91992188,175.00486475)
\curveto(36.91992188,174.5478335)(37.08691406,174.18748194)(37.42089844,173.92381006)
\curveto(37.75488281,173.66013819)(38.15039062,173.52830225)(38.60742188,173.52830225)
\curveto(39.1640625,173.52830225)(39.703125,173.6572085)(40.22460938,173.915021)
\curveto(41.10351562,174.34275538)(41.54296875,175.04295069)(41.54296875,176.01560694)
\lineto(41.54296875,177.290021)
\curveto(41.34960938,177.16697413)(41.10058594,177.06443506)(40.79589844,176.98240381)
\curveto(40.49121094,176.90037256)(40.19238281,176.84177881)(39.89941406,176.80662256)
\lineto(38.94140625,176.68357569)
\curveto(38.3671875,176.60740381)(37.93652344,176.48728663)(37.64941406,176.32322413)
\curveto(37.16308594,176.0478335)(36.91992188,175.60838038)(36.91992188,175.00486475)
\closepath
\moveto(40.75195312,178.2040835)
\curveto(41.11523438,178.2509585)(41.35839844,178.40330225)(41.48144531,178.66111475)
\curveto(41.55175781,178.80173975)(41.58691406,179.00388819)(41.58691406,179.26756006)
\curveto(41.58691406,179.80662256)(41.39355469,180.196271)(41.00683594,180.43650538)
\curveto(40.62597656,180.68259913)(40.078125,180.805646)(39.36328125,180.805646)
\curveto(38.53710938,180.805646)(37.95117188,180.58298975)(37.60546875,180.13767725)
\curveto(37.41210938,179.8915835)(37.28613281,179.52537256)(37.22753906,179.03904444)
\lineto(35.75097656,179.03904444)
\curveto(35.78027344,180.19920069)(36.15527344,181.00486475)(36.87597656,181.45603663)
\curveto(37.60253906,181.91306788)(38.44335938,182.1415835)(39.3984375,182.1415835)
\curveto(40.50585938,182.1415835)(41.40527344,181.930646)(42.09667969,181.508771)
\curveto(42.78222656,181.086896)(43.125,180.430646)(43.125,179.540021)
\lineto(43.125,174.11716944)
\curveto(43.125,173.95310694)(43.15722656,173.821271)(43.22167969,173.72166163)
\curveto(43.29199219,173.62205225)(43.43554688,173.57224756)(43.65234375,173.57224756)
\curveto(43.72265625,173.57224756)(43.80175781,173.57517725)(43.88964844,173.58103663)
\curveto(43.97753906,173.59275538)(44.07128906,173.60740381)(44.17089844,173.62498194)
\lineto(44.17089844,172.45603663)
\curveto(43.92480469,172.38572413)(43.73730469,172.34177881)(43.60839844,172.32420069)
\curveto(43.47949219,172.30662256)(43.30371094,172.2978335)(43.08105469,172.2978335)
\curveto(42.53613281,172.2978335)(42.140625,172.49119288)(41.89453125,172.87791163)
\curveto(41.765625,173.08298975)(41.67480469,173.37302881)(41.62207031,173.74802881)
\curveto(41.29980469,173.32615381)(40.83691406,172.95994288)(40.23339844,172.649396)
\curveto(39.62988281,172.33884913)(38.96484375,172.18357569)(38.23828125,172.18357569)
\curveto(37.36523438,172.18357569)(36.65039062,172.44724756)(36.09375,172.97459131)
\curveto(35.54296875,173.50779444)(35.26757812,174.1728335)(35.26757812,174.9697085)
\curveto(35.26757812,175.84275538)(35.54003906,176.51951319)(36.08496094,176.99998194)
\curveto(36.62988281,177.48045069)(37.34472656,177.77634913)(38.22949219,177.88767725)
\lineto(40.75195312,178.2040835)
\closepath
\moveto(39.44238281,182.1415835)
\lineto(39.44238281,182.1415835)
\closepath
}
}
{
\newrgbcolor{curcolor}{0 0 0}
\pscustom[linestyle=none,fillstyle=solid,fillcolor=curcolor]
{
\newpath
\moveto(45.77050781,181.91306788)
\lineto(47.2734375,181.91306788)
\lineto(47.2734375,180.28709131)
\curveto(47.39648438,180.60349756)(47.69824219,180.98728663)(48.17871094,181.4384585)
\curveto(48.65917969,181.89548975)(49.21289062,182.12400538)(49.83984375,182.12400538)
\curveto(49.86914062,182.12400538)(49.91894531,182.12107569)(49.98925781,182.11521631)
\curveto(50.05957031,182.10935694)(50.1796875,182.09763819)(50.34960938,182.08006006)
\lineto(50.34960938,180.41013819)
\curveto(50.25585938,180.42771631)(50.16796875,180.43943506)(50.0859375,180.44529444)
\curveto(50.00976562,180.45115381)(49.92480469,180.4540835)(49.83105469,180.4540835)
\curveto(49.03417969,180.4540835)(48.421875,180.196271)(47.99414062,179.680646)
\curveto(47.56640625,179.17088038)(47.35253906,178.58201319)(47.35253906,177.91404444)
\lineto(47.35253906,172.49998194)
\lineto(45.77050781,172.49998194)
\lineto(45.77050781,181.91306788)
\closepath
}
}
{
\newrgbcolor{curcolor}{0 0 0}
\pscustom[linestyle=none,fillstyle=solid,fillcolor=curcolor]
{
\newpath
\moveto(57.61816406,181.91306788)
\lineto(59.3671875,181.91306788)
\curveto(59.14453125,181.30955225)(58.64941406,179.93259913)(57.88183594,177.7822085)
\curveto(57.30761719,176.165021)(56.82714844,174.84666163)(56.44042969,173.82713038)
\curveto(55.52636719,171.42478663)(54.88183594,169.95994288)(54.50683594,169.43259913)
\curveto(54.13183594,168.90525538)(53.48730469,168.6415835)(52.57324219,168.6415835)
\curveto(52.35058594,168.6415835)(52.17773438,168.65037256)(52.0546875,168.66795069)
\curveto(51.9375,168.68552881)(51.79101562,168.71775538)(51.61523438,168.76463038)
\lineto(51.61523438,170.20603663)
\curveto(51.890625,170.12986475)(52.08984375,170.08298975)(52.21289062,170.06541163)
\curveto(52.3359375,170.0478335)(52.44433594,170.03904444)(52.53808594,170.03904444)
\curveto(52.83105469,170.03904444)(53.04492188,170.08884913)(53.1796875,170.1884585)
\curveto(53.3203125,170.2822085)(53.4375,170.399396)(53.53125,170.540021)
\curveto(53.56054688,170.586896)(53.66601562,170.82713038)(53.84765625,171.26072413)
\curveto(54.02929688,171.69431788)(54.16113281,172.0165835)(54.24316406,172.227521)
\lineto(50.76269531,181.91306788)
\lineto(52.55566406,181.91306788)
\lineto(55.078125,174.24900538)
\lineto(57.61816406,181.91306788)
\closepath
\moveto(55.06933594,182.1415835)
\lineto(55.06933594,182.1415835)
\closepath
}
}
{
\newrgbcolor{curcolor}{0 0 0}
\pscustom[linewidth=1,linecolor=curcolor]
{
\newpath
\moveto(140.5,192.50001246)
\lineto(260.5,192.50001246)
\lineto(260.5,162.50001246)
\lineto(140.5,162.50001246)
\closepath
}
}
{
\newrgbcolor{curcolor}{0 0 0}
\pscustom[linestyle=none,fillstyle=solid,fillcolor=curcolor]
{
\newpath
\moveto(151.9765625,184.54099756)
\lineto(153.57617188,184.54099756)
\lineto(153.57617188,181.91306788)
\lineto(155.07910156,181.91306788)
\lineto(155.07910156,180.62107569)
\lineto(153.57617188,180.62107569)
\lineto(153.57617188,174.477521)
\curveto(153.57617188,174.149396)(153.6875,173.92966944)(153.91015625,173.81834131)
\curveto(154.03320312,173.75388819)(154.23828125,173.72166163)(154.52539062,173.72166163)
\lineto(154.77148438,173.72166163)
\curveto(154.859375,173.727521)(154.96191406,173.73631006)(155.07910156,173.74802881)
\lineto(155.07910156,172.49998194)
\curveto(154.89746094,172.44724756)(154.70703125,172.40916163)(154.5078125,172.38572413)
\curveto(154.31445312,172.36228663)(154.10351562,172.35056788)(153.875,172.35056788)
\curveto(153.13671875,172.35056788)(152.63574219,172.53806788)(152.37207031,172.91306788)
\curveto(152.10839844,173.29392725)(151.9765625,173.78611475)(151.9765625,174.38963038)
\lineto(151.9765625,180.62107569)
\lineto(150.70214844,180.62107569)
\lineto(150.70214844,181.91306788)
\lineto(151.9765625,181.91306788)
\lineto(151.9765625,184.54099756)
\closepath
}
}
{
\newrgbcolor{curcolor}{0 0 0}
\pscustom[linestyle=none,fillstyle=solid,fillcolor=curcolor]
{
\newpath
\moveto(160.60742188,182.12400538)
\curveto(161.27539062,182.12400538)(161.92285156,181.96580225)(162.54980469,181.649396)
\curveto(163.17675781,181.33884913)(163.65429688,180.93455225)(163.98242188,180.43650538)
\curveto(164.29882812,179.961896)(164.50976562,179.40818506)(164.61523438,178.77537256)
\curveto(164.70898438,178.34177881)(164.75585938,177.65037256)(164.75585938,176.70115381)
\lineto(157.85644531,176.70115381)
\curveto(157.88574219,175.74607569)(158.11132812,174.97849756)(158.53320312,174.39841944)
\curveto(158.95507812,173.82420069)(159.60839844,173.53709131)(160.49316406,173.53709131)
\curveto(161.31933594,173.53709131)(161.97851562,173.80955225)(162.47070312,174.35447413)
\curveto(162.75195312,174.67088038)(162.95117188,175.03709131)(163.06835938,175.45310694)
\lineto(164.62402344,175.45310694)
\curveto(164.58300781,175.10740381)(164.4453125,174.72068506)(164.2109375,174.29295069)
\curveto(163.98242188,173.87107569)(163.72460938,173.52537256)(163.4375,173.25584131)
\curveto(162.95703125,172.78709131)(162.36230469,172.47068506)(161.65332031,172.30662256)
\curveto(161.27246094,172.21287256)(160.84179688,172.16599756)(160.36132812,172.16599756)
\curveto(159.18945312,172.16599756)(158.19628906,172.59080225)(157.38183594,173.44041163)
\curveto(156.56738281,174.29588038)(156.16015625,175.49119288)(156.16015625,177.02634913)
\curveto(156.16015625,178.53806788)(156.5703125,179.76560694)(157.390625,180.70896631)
\curveto(158.2109375,181.65232569)(159.28320312,182.12400538)(160.60742188,182.12400538)
\closepath
\moveto(163.12988281,177.95798975)
\curveto(163.06542969,178.64353663)(162.91601562,179.19138819)(162.68164062,179.60154444)
\curveto(162.24804688,180.36326319)(161.52441406,180.74412256)(160.51074219,180.74412256)
\curveto(159.78417969,180.74412256)(159.17480469,180.48045069)(158.68261719,179.95310694)
\curveto(158.19042969,179.43162256)(157.9296875,178.7665835)(157.90039062,177.95798975)
\lineto(163.12988281,177.95798975)
\closepath
\moveto(160.45800781,182.1415835)
\lineto(160.45800781,182.1415835)
\closepath
}
}
{
\newrgbcolor{curcolor}{0 0 0}
\pscustom[linestyle=none,fillstyle=solid,fillcolor=curcolor]
{
\newpath
\moveto(165.81054688,181.91306788)
\lineto(167.85839844,181.91306788)
\lineto(170.02050781,178.59959131)
\lineto(172.20898438,181.91306788)
\lineto(174.13378906,181.86912256)
\lineto(170.9609375,177.32517725)
\lineto(174.27441406,172.49998194)
\lineto(172.25292969,172.49998194)
\lineto(169.91503906,176.03318506)
\lineto(167.64746094,172.49998194)
\lineto(165.64355469,172.49998194)
\lineto(168.95703125,177.32517725)
\lineto(165.81054688,181.91306788)
\closepath
}
}
{
\newrgbcolor{curcolor}{0 0 0}
\pscustom[linestyle=none,fillstyle=solid,fillcolor=curcolor]
{
\newpath
\moveto(176.0234375,184.54099756)
\lineto(177.62304688,184.54099756)
\lineto(177.62304688,181.91306788)
\lineto(179.12597656,181.91306788)
\lineto(179.12597656,180.62107569)
\lineto(177.62304688,180.62107569)
\lineto(177.62304688,174.477521)
\curveto(177.62304688,174.149396)(177.734375,173.92966944)(177.95703125,173.81834131)
\curveto(178.08007812,173.75388819)(178.28515625,173.72166163)(178.57226562,173.72166163)
\lineto(178.81835938,173.72166163)
\curveto(178.90625,173.727521)(179.00878906,173.73631006)(179.12597656,173.74802881)
\lineto(179.12597656,172.49998194)
\curveto(178.94433594,172.44724756)(178.75390625,172.40916163)(178.5546875,172.38572413)
\curveto(178.36132812,172.36228663)(178.15039062,172.35056788)(177.921875,172.35056788)
\curveto(177.18359375,172.35056788)(176.68261719,172.53806788)(176.41894531,172.91306788)
\curveto(176.15527344,173.29392725)(176.0234375,173.78611475)(176.0234375,174.38963038)
\lineto(176.0234375,180.62107569)
\lineto(174.74902344,180.62107569)
\lineto(174.74902344,181.91306788)
\lineto(176.0234375,181.91306788)
\lineto(176.0234375,184.54099756)
\closepath
}
}
{
\newrgbcolor{curcolor}{0 0 0}
\pscustom[linewidth=1,linecolor=curcolor]
{
\newpath
\moveto(431.78541,190.78746236)
\lineto(551.78541,190.78746236)
\lineto(551.78541,160.78746236)
\lineto(431.78541,160.78746236)
\closepath
}
}
{
\newrgbcolor{curcolor}{0 0 0}
\pscustom[linestyle=none,fillstyle=solid,fillcolor=curcolor]
{
\newpath
\moveto(442.94556625,180.15657246)
\lineto(444.55396469,180.15657246)
\lineto(444.55396469,170.78743184)
\lineto(442.94556625,170.78743184)
\lineto(442.94556625,180.15657246)
\closepath
\moveto(442.94556625,183.69856465)
\lineto(444.55396469,183.69856465)
\lineto(444.55396469,181.9055959)
\lineto(442.94556625,181.9055959)
\lineto(442.94556625,183.69856465)
\closepath
}
}
{
\newrgbcolor{curcolor}{0 0 0}
\pscustom[linestyle=none,fillstyle=solid,fillcolor=curcolor]
{
\newpath
\moveto(446.95337875,180.20051778)
\lineto(448.45630844,180.20051778)
\lineto(448.45630844,178.86458028)
\curveto(448.90162094,179.41536153)(449.37330063,179.81086934)(449.8713475,180.05110371)
\curveto(450.36939438,180.29133809)(450.92310531,180.41145528)(451.53248031,180.41145528)
\curveto(452.86841781,180.41145528)(453.77076156,179.94563496)(454.23951156,179.01399434)
\curveto(454.49732406,178.50422871)(454.62623031,177.77473653)(454.62623031,176.82551778)
\lineto(454.62623031,170.78743184)
\lineto(453.01783188,170.78743184)
\lineto(453.01783188,176.72004903)
\curveto(453.01783188,177.29426778)(452.93287094,177.7571584)(452.76294906,178.1087209)
\curveto(452.48169906,178.6946584)(451.97193344,178.98762715)(451.23365219,178.98762715)
\curveto(450.85865219,178.98762715)(450.551035,178.94954121)(450.31080063,178.87336934)
\curveto(449.87720688,178.74446309)(449.4963475,178.48665059)(449.1682225,178.09993184)
\curveto(448.90455063,177.78938496)(448.73169906,177.46711934)(448.64966781,177.13313496)
\curveto(448.57349594,176.80500996)(448.53541,176.33333028)(448.53541,175.7180959)
\lineto(448.53541,170.78743184)
\lineto(446.95337875,170.78743184)
\lineto(446.95337875,180.20051778)
\closepath
\moveto(450.67115219,180.4290334)
\lineto(450.67115219,180.4290334)
\closepath
}
}
{
\newrgbcolor{curcolor}{0 0 0}
\pscustom[linestyle=none,fillstyle=solid,fillcolor=curcolor]
{
\newpath
\moveto(457.28931625,182.82844746)
\lineto(458.88892563,182.82844746)
\lineto(458.88892563,180.20051778)
\lineto(460.39185531,180.20051778)
\lineto(460.39185531,178.90852559)
\lineto(458.88892563,178.90852559)
\lineto(458.88892563,172.7649709)
\curveto(458.88892563,172.4368459)(459.00025375,172.21711934)(459.22291,172.10579121)
\curveto(459.34595688,172.04133809)(459.551035,172.00911153)(459.83814438,172.00911153)
\lineto(460.08423813,172.00911153)
\curveto(460.17212875,172.0149709)(460.27466781,172.02375996)(460.39185531,172.03547871)
\lineto(460.39185531,170.78743184)
\curveto(460.21021469,170.73469746)(460.019785,170.69661153)(459.82056625,170.67317403)
\curveto(459.62720688,170.64973653)(459.41626938,170.63801778)(459.18775375,170.63801778)
\curveto(458.4494725,170.63801778)(457.94849594,170.82551778)(457.68482406,171.20051778)
\curveto(457.42115219,171.58137715)(457.28931625,172.07356465)(457.28931625,172.67708028)
\lineto(457.28931625,178.90852559)
\lineto(456.01490219,178.90852559)
\lineto(456.01490219,180.20051778)
\lineto(457.28931625,180.20051778)
\lineto(457.28931625,182.82844746)
\closepath
}
}
{
\newrgbcolor{curcolor}{0 0 0}
\pscustom[linestyle=none,fillstyle=solid,fillcolor=curcolor]
{
\newpath
\moveto(465.92017563,180.41145528)
\curveto(466.58814438,180.41145528)(467.23560531,180.25325215)(467.86255844,179.9368459)
\curveto(468.48951156,179.62629903)(468.96705063,179.22200215)(469.29517563,178.72395528)
\curveto(469.61158188,178.2493459)(469.82251938,177.69563496)(469.92798813,177.06282246)
\curveto(470.02173813,176.62922871)(470.06861313,175.93782246)(470.06861313,174.98860371)
\lineto(463.16919906,174.98860371)
\curveto(463.19849594,174.03352559)(463.42408188,173.26594746)(463.84595688,172.68586934)
\curveto(464.26783188,172.11165059)(464.92115219,171.82454121)(465.80591781,171.82454121)
\curveto(466.63208969,171.82454121)(467.29126938,172.09700215)(467.78345688,172.64192403)
\curveto(468.06470688,172.95833028)(468.26392563,173.32454121)(468.38111313,173.74055684)
\lineto(469.93677719,173.74055684)
\curveto(469.89576156,173.39485371)(469.75806625,173.00813496)(469.52369125,172.58040059)
\curveto(469.29517563,172.15852559)(469.03736313,171.81282246)(468.75025375,171.54329121)
\curveto(468.269785,171.07454121)(467.67505844,170.75813496)(466.96607406,170.59407246)
\curveto(466.58521469,170.50032246)(466.15455063,170.45344746)(465.67408188,170.45344746)
\curveto(464.50220688,170.45344746)(463.50904281,170.87825215)(462.69458969,171.72786153)
\curveto(461.88013656,172.58333028)(461.47291,173.77864278)(461.47291,175.31379903)
\curveto(461.47291,176.82551778)(461.88306625,178.05305684)(462.70337875,178.99641621)
\curveto(463.52369125,179.93977559)(464.59595688,180.41145528)(465.92017563,180.41145528)
\closepath
\moveto(468.44263656,176.24543965)
\curveto(468.37818344,176.93098653)(468.22876938,177.47883809)(467.99439438,177.88899434)
\curveto(467.56080063,178.65071309)(466.83716781,179.03157246)(465.82349594,179.03157246)
\curveto(465.09693344,179.03157246)(464.48755844,178.76790059)(463.99537094,178.24055684)
\curveto(463.50318344,177.71907246)(463.24244125,177.0540334)(463.21314438,176.24543965)
\lineto(468.44263656,176.24543965)
\closepath
\moveto(465.77076156,180.4290334)
\lineto(465.77076156,180.4290334)
\closepath
}
}
{
\newrgbcolor{curcolor}{0 0 0}
\pscustom[linestyle=none,fillstyle=solid,fillcolor=curcolor]
{
\newpath
\moveto(472.06373031,180.20051778)
\lineto(473.56666,180.20051778)
\lineto(473.56666,178.57454121)
\curveto(473.68970688,178.89094746)(473.99146469,179.27473653)(474.47193344,179.7259084)
\curveto(474.95240219,180.18293965)(475.50611313,180.41145528)(476.13306625,180.41145528)
\curveto(476.16236313,180.41145528)(476.21216781,180.40852559)(476.28248031,180.40266621)
\curveto(476.35279281,180.39680684)(476.47291,180.38508809)(476.64283188,180.36750996)
\lineto(476.64283188,178.69758809)
\curveto(476.54908188,178.71516621)(476.46119125,178.72688496)(476.37916,178.73274434)
\curveto(476.30298813,178.73860371)(476.21802719,178.7415334)(476.12427719,178.7415334)
\curveto(475.32740219,178.7415334)(474.7150975,178.4837209)(474.28736313,177.9680959)
\curveto(473.85962875,177.45833028)(473.64576156,176.86946309)(473.64576156,176.20149434)
\lineto(473.64576156,170.78743184)
\lineto(472.06373031,170.78743184)
\lineto(472.06373031,180.20051778)
\closepath
}
}
{
\newrgbcolor{curcolor}{0 0 0}
\pscustom[linestyle=none,fillstyle=solid,fillcolor=curcolor]
{
\newpath
\moveto(478.03150375,180.20051778)
\lineto(479.53443344,180.20051778)
\lineto(479.53443344,178.86458028)
\curveto(479.97974594,179.41536153)(480.45142563,179.81086934)(480.9494725,180.05110371)
\curveto(481.44751938,180.29133809)(482.00123031,180.41145528)(482.61060531,180.41145528)
\curveto(483.94654281,180.41145528)(484.84888656,179.94563496)(485.31763656,179.01399434)
\curveto(485.57544906,178.50422871)(485.70435531,177.77473653)(485.70435531,176.82551778)
\lineto(485.70435531,170.78743184)
\lineto(484.09595688,170.78743184)
\lineto(484.09595688,176.72004903)
\curveto(484.09595688,177.29426778)(484.01099594,177.7571584)(483.84107406,178.1087209)
\curveto(483.55982406,178.6946584)(483.05005844,178.98762715)(482.31177719,178.98762715)
\curveto(481.93677719,178.98762715)(481.62916,178.94954121)(481.38892563,178.87336934)
\curveto(480.95533188,178.74446309)(480.5744725,178.48665059)(480.2463475,178.09993184)
\curveto(479.98267563,177.78938496)(479.80982406,177.46711934)(479.72779281,177.13313496)
\curveto(479.65162094,176.80500996)(479.613535,176.33333028)(479.613535,175.7180959)
\lineto(479.613535,170.78743184)
\lineto(478.03150375,170.78743184)
\lineto(478.03150375,180.20051778)
\closepath
\moveto(481.74927719,180.4290334)
\lineto(481.74927719,180.4290334)
\closepath
}
}
{
\newrgbcolor{curcolor}{0 0 0}
\pscustom[linestyle=none,fillstyle=solid,fillcolor=curcolor]
{
\newpath
\moveto(489.26392563,173.29231465)
\curveto(489.26392563,172.8352834)(489.43091781,172.47493184)(489.76490219,172.21125996)
\curveto(490.09888656,171.94758809)(490.49439438,171.81575215)(490.95142563,171.81575215)
\curveto(491.50806625,171.81575215)(492.04712875,171.9446584)(492.56861313,172.2024709)
\curveto(493.44751938,172.63020528)(493.8869725,173.33040059)(493.8869725,174.30305684)
\lineto(493.8869725,175.5774709)
\curveto(493.69361313,175.45442403)(493.44458969,175.35188496)(493.13990219,175.26985371)
\curveto(492.83521469,175.18782246)(492.53638656,175.12922871)(492.24341781,175.09407246)
\lineto(491.28541,174.97102559)
\curveto(490.71119125,174.89485371)(490.28052719,174.77473653)(489.99341781,174.61067403)
\curveto(489.50708969,174.3352834)(489.26392563,173.89583028)(489.26392563,173.29231465)
\closepath
\moveto(493.09595688,176.4915334)
\curveto(493.45923813,176.5384084)(493.70240219,176.69075215)(493.82544906,176.94856465)
\curveto(493.89576156,177.08918965)(493.93091781,177.29133809)(493.93091781,177.55500996)
\curveto(493.93091781,178.09407246)(493.73755844,178.4837209)(493.35083969,178.72395528)
\curveto(492.96998031,178.97004903)(492.42212875,179.0930959)(491.707285,179.0930959)
\curveto(490.88111313,179.0930959)(490.29517563,178.87043965)(489.9494725,178.42512715)
\curveto(489.75611313,178.1790334)(489.63013656,177.81282246)(489.57154281,177.32649434)
\lineto(488.09498031,177.32649434)
\curveto(488.12427719,178.48665059)(488.49927719,179.29231465)(489.21998031,179.74348653)
\curveto(489.94654281,180.20051778)(490.78736313,180.4290334)(491.74244125,180.4290334)
\curveto(492.84986313,180.4290334)(493.74927719,180.2180959)(494.44068344,179.7962209)
\curveto(495.12623031,179.3743459)(495.46900375,178.7180959)(495.46900375,177.8274709)
\lineto(495.46900375,172.40461934)
\curveto(495.46900375,172.24055684)(495.50123031,172.1087209)(495.56568344,172.00911153)
\curveto(495.63599594,171.90950215)(495.77955063,171.85969746)(495.9963475,171.85969746)
\curveto(496.06666,171.85969746)(496.14576156,171.86262715)(496.23365219,171.86848653)
\curveto(496.32154281,171.88020528)(496.41529281,171.89485371)(496.51490219,171.91243184)
\lineto(496.51490219,170.74348653)
\curveto(496.26880844,170.67317403)(496.08130844,170.62922871)(495.95240219,170.61165059)
\curveto(495.82349594,170.59407246)(495.64771469,170.5852834)(495.42505844,170.5852834)
\curveto(494.88013656,170.5852834)(494.48462875,170.77864278)(494.238535,171.16536153)
\curveto(494.10962875,171.37043965)(494.01880844,171.66047871)(493.96607406,172.03547871)
\curveto(493.64380844,171.61360371)(493.18091781,171.24739278)(492.57740219,170.9368459)
\curveto(491.97388656,170.62629903)(491.3088475,170.47102559)(490.582285,170.47102559)
\curveto(489.70923813,170.47102559)(488.99439438,170.73469746)(488.43775375,171.26204121)
\curveto(487.8869725,171.79524434)(487.61158188,172.4602834)(487.61158188,173.2571584)
\curveto(487.61158188,174.13020528)(487.88404281,174.80696309)(488.42896469,175.28743184)
\curveto(488.97388656,175.76790059)(489.68873031,176.06379903)(490.57349594,176.17512715)
\lineto(493.09595688,176.4915334)
\closepath
\moveto(491.78638656,180.4290334)
\lineto(491.78638656,180.4290334)
\closepath
}
}
{
\newrgbcolor{curcolor}{0 0 0}
\pscustom[linestyle=none,fillstyle=solid,fillcolor=curcolor]
{
\newpath
\moveto(498.11451156,183.69856465)
\lineto(499.69654281,183.69856465)
\lineto(499.69654281,170.78743184)
\lineto(498.11451156,170.78743184)
\lineto(498.11451156,183.69856465)
\closepath
}
}
{
\newrgbcolor{curcolor}{0 0 0}
\pscustom[linewidth=1,linecolor=curcolor]
{
\newpath
\moveto(291.78342,190.78746206)
\lineto(411.78342,190.78746206)
\lineto(411.78342,160.78746206)
\lineto(291.78342,160.78746206)
\closepath
}
}
{
\newrgbcolor{curcolor}{0 0 0}
\pscustom[linestyle=none,fillstyle=solid,fillcolor=curcolor]
{
\newpath
\moveto(303.88400594,173.74055654)
\curveto(303.93088094,173.21321279)(304.06271687,172.80891591)(304.27951375,172.52766591)
\curveto(304.67795125,172.01790029)(305.3693575,171.76301748)(306.3537325,171.76301748)
\curveto(306.93967,171.76301748)(307.455295,171.88899404)(307.9006075,172.14094716)
\curveto(308.34592,172.39875966)(308.56857625,172.79426748)(308.56857625,173.3274706)
\curveto(308.56857625,173.73176748)(308.38986531,174.03938466)(308.03244344,174.25032216)
\curveto(307.80392781,174.37922841)(307.35275594,174.52864248)(306.67892781,174.69856435)
\lineto(305.42209187,175.0149706)
\curveto(304.6193575,175.21418935)(304.02756062,175.4368456)(303.64670125,175.68293935)
\curveto(302.96701375,176.11067373)(302.62717,176.7024706)(302.62717,177.45832998)
\curveto(302.62717,178.34895498)(302.94650594,179.0696581)(303.58517781,179.62043935)
\curveto(304.22970906,180.1712206)(305.09396687,180.44661123)(306.17795125,180.44661123)
\curveto(307.59592,180.44661123)(308.61838094,180.0305956)(309.24533406,179.19856435)
\curveto(309.63791219,178.6712206)(309.82834187,178.10286123)(309.81662312,177.49348623)
\lineto(308.3224825,177.49348623)
\curveto(308.29318562,177.8509081)(308.16720906,178.17610341)(307.94455281,178.46907216)
\curveto(307.58127156,178.88508779)(306.95138875,179.0930956)(306.05490437,179.0930956)
\curveto(305.45724812,179.0930956)(305.00314656,178.97883779)(304.69259969,178.75032216)
\curveto(304.38791219,178.52180654)(304.23556844,178.22004873)(304.23556844,177.84504873)
\curveto(304.23556844,177.43489248)(304.43771687,177.10676748)(304.84201375,176.86067373)
\curveto(305.07638875,176.71418935)(305.42209187,176.5852831)(305.87912312,176.47395498)
\lineto(306.92502156,176.21907216)
\curveto(308.06174031,175.94368154)(308.82345906,175.67707998)(309.21017781,175.41926748)
\curveto(309.82541219,175.0149706)(310.13302937,174.37922841)(310.13302937,173.51204091)
\curveto(310.13302937,172.67415029)(309.81369344,171.95051748)(309.17502156,171.34114248)
\curveto(308.54220906,170.73176748)(307.57541219,170.42707998)(306.27463094,170.42707998)
\curveto(304.87424031,170.42707998)(303.88107625,170.74348623)(303.29513875,171.37629873)
\curveto(302.71506062,172.0149706)(302.40451375,172.80305654)(302.36349812,173.74055654)
\lineto(303.88400594,173.74055654)
\closepath
\moveto(306.22189656,180.4290331)
\lineto(306.22189656,180.4290331)
\closepath
}
}
{
\newrgbcolor{curcolor}{0 0 0}
\pscustom[linestyle=none,fillstyle=solid,fillcolor=curcolor]
{
\newpath
\moveto(312.75217,180.20051748)
\lineto(315.26584187,172.53645498)
\lineto(317.89377156,180.20051748)
\lineto(319.62521687,180.20051748)
\lineto(316.07443562,170.78743154)
\lineto(314.38693562,170.78743154)
\lineto(310.91525594,180.20051748)
\lineto(312.75217,180.20051748)
\closepath
}
}
{
\newrgbcolor{curcolor}{0 0 0}
\pscustom[linestyle=none,fillstyle=solid,fillcolor=curcolor]
{
\newpath
\moveto(324.30099812,180.36750966)
\curveto(325.03927937,180.36750966)(325.68381062,180.18586904)(326.23459187,179.82258779)
\curveto(326.53342,179.61750966)(326.8381075,179.31868154)(327.14865437,178.92610341)
\lineto(327.14865437,180.11262685)
\lineto(328.60763875,180.11262685)
\lineto(328.60763875,171.55207998)
\curveto(328.60763875,170.35676748)(328.4318575,169.4134081)(328.080295,168.72200185)
\curveto(327.424045,167.4446581)(326.18478719,166.80598623)(324.36252156,166.80598623)
\curveto(323.34884969,166.80598623)(322.49631062,167.03450185)(321.80490437,167.4915331)
\curveto(321.11349812,167.94270498)(320.72677937,168.65168935)(320.64474812,169.61848623)
\lineto(322.25314656,169.61848623)
\curveto(322.32931844,169.19661123)(322.48166219,168.87141591)(322.71017781,168.64290029)
\curveto(323.06759969,168.29133779)(323.63009969,168.11555654)(324.39767781,168.11555654)
\curveto(325.61056844,168.11555654)(326.40451375,168.54329091)(326.77951375,169.39875966)
\curveto(327.00217,169.90266591)(327.10470906,170.80207998)(327.08713094,172.09700185)
\curveto(326.77072469,171.6165331)(326.38986531,171.25911123)(325.94455281,171.02473623)
\curveto(325.49924031,170.79036123)(324.91037312,170.67317373)(324.17795125,170.67317373)
\curveto(323.15842,170.67317373)(322.26486531,171.03352529)(321.49728719,171.75422841)
\curveto(320.73556844,172.48079091)(320.35470906,173.6790331)(320.35470906,175.34895498)
\curveto(320.35470906,176.92512685)(320.73849812,178.1555956)(321.50607625,179.04036123)
\curveto(322.27951375,179.92512685)(323.21115437,180.36750966)(324.30099812,180.36750966)
\closepath
\moveto(327.14865437,175.53352529)
\curveto(327.14865437,176.69954091)(326.90842,177.56379873)(326.42795125,178.12629873)
\curveto(325.9474825,178.68879873)(325.33517781,178.97004873)(324.59103719,178.97004873)
\curveto(323.47775594,178.97004873)(322.71603719,178.44856435)(322.30588094,177.4055956)
\curveto(322.08908406,176.84895498)(321.98068562,176.11946279)(321.98068562,175.21711904)
\curveto(321.98068562,174.15657216)(322.19455281,173.34797841)(322.62228719,172.79133779)
\curveto(323.05588094,172.24055654)(323.63595906,171.96516591)(324.36252156,171.96516591)
\curveto(325.49924031,171.96516591)(326.299045,172.47786123)(326.76193562,173.50325185)
\curveto(327.01974812,174.08332998)(327.14865437,174.76008779)(327.14865437,175.53352529)
\closepath
\moveto(324.48556844,180.4290331)
\lineto(324.48556844,180.4290331)
\closepath
}
}
{
\newrgbcolor{curcolor}{0 0 0}
\pscustom[linewidth=1,linecolor=curcolor]
{
\newpath
\moveto(112.5,110.50001246)
\lineto(232.5,110.50001246)
\lineto(232.5,80.50001246)
\lineto(112.5,80.50001246)
\closepath
}
}
{
\newrgbcolor{curcolor}{0 0 0}
\pscustom[linestyle=none,fillstyle=solid,fillcolor=curcolor]
{
\newpath
\moveto(131.66015625,105.45506006)
\lineto(133.2421875,105.45506006)
\lineto(133.2421875,100.63865381)
\curveto(133.6171875,101.11326319)(133.95410156,101.44724756)(134.25292969,101.64060694)
\curveto(134.76269531,101.97459131)(135.3984375,102.1415835)(136.16015625,102.1415835)
\curveto(137.52539062,102.1415835)(138.45117188,101.66404444)(138.9375,100.70896631)
\curveto(139.20117188,100.18748194)(139.33300781,99.46384913)(139.33300781,98.53806788)
\lineto(139.33300781,92.49998194)
\lineto(137.70703125,92.49998194)
\lineto(137.70703125,98.43259913)
\curveto(137.70703125,99.12400538)(137.61914062,99.63084131)(137.44335938,99.95310694)
\curveto(137.15625,100.46873194)(136.6171875,100.72654444)(135.82617188,100.72654444)
\curveto(135.16992188,100.72654444)(134.57519531,100.5009585)(134.04199219,100.04978663)
\curveto(133.50878906,99.59861475)(133.2421875,98.74607569)(133.2421875,97.49216944)
\lineto(133.2421875,92.49998194)
\lineto(131.66015625,92.49998194)
\lineto(131.66015625,105.45506006)
\closepath
}
}
{
\newrgbcolor{curcolor}{0 0 0}
\pscustom[linestyle=none,fillstyle=solid,fillcolor=curcolor]
{
\newpath
\moveto(141.99609375,104.54099756)
\lineto(143.59570312,104.54099756)
\lineto(143.59570312,101.91306788)
\lineto(145.09863281,101.91306788)
\lineto(145.09863281,100.62107569)
\lineto(143.59570312,100.62107569)
\lineto(143.59570312,94.477521)
\curveto(143.59570312,94.149396)(143.70703125,93.92966944)(143.9296875,93.81834131)
\curveto(144.05273438,93.75388819)(144.2578125,93.72166163)(144.54492188,93.72166163)
\lineto(144.79101562,93.72166163)
\curveto(144.87890625,93.727521)(144.98144531,93.73631006)(145.09863281,93.74802881)
\lineto(145.09863281,92.49998194)
\curveto(144.91699219,92.44724756)(144.7265625,92.40916163)(144.52734375,92.38572413)
\curveto(144.33398438,92.36228663)(144.12304688,92.35056788)(143.89453125,92.35056788)
\curveto(143.15625,92.35056788)(142.65527344,92.53806788)(142.39160156,92.91306788)
\curveto(142.12792969,93.29392725)(141.99609375,93.78611475)(141.99609375,94.38963038)
\lineto(141.99609375,100.62107569)
\lineto(140.72167969,100.62107569)
\lineto(140.72167969,101.91306788)
\lineto(141.99609375,101.91306788)
\lineto(141.99609375,104.54099756)
\closepath
}
}
{
\newrgbcolor{curcolor}{0 0 0}
\pscustom[linestyle=none,fillstyle=solid,fillcolor=curcolor]
{
\newpath
\moveto(147.0234375,104.54099756)
\lineto(148.62304688,104.54099756)
\lineto(148.62304688,101.91306788)
\lineto(150.12597656,101.91306788)
\lineto(150.12597656,100.62107569)
\lineto(148.62304688,100.62107569)
\lineto(148.62304688,94.477521)
\curveto(148.62304688,94.149396)(148.734375,93.92966944)(148.95703125,93.81834131)
\curveto(149.08007812,93.75388819)(149.28515625,93.72166163)(149.57226562,93.72166163)
\lineto(149.81835938,93.72166163)
\curveto(149.90625,93.727521)(150.00878906,93.73631006)(150.12597656,93.74802881)
\lineto(150.12597656,92.49998194)
\curveto(149.94433594,92.44724756)(149.75390625,92.40916163)(149.5546875,92.38572413)
\curveto(149.36132812,92.36228663)(149.15039062,92.35056788)(148.921875,92.35056788)
\curveto(148.18359375,92.35056788)(147.68261719,92.53806788)(147.41894531,92.91306788)
\curveto(147.15527344,93.29392725)(147.0234375,93.78611475)(147.0234375,94.38963038)
\lineto(147.0234375,100.62107569)
\lineto(145.74902344,100.62107569)
\lineto(145.74902344,101.91306788)
\lineto(147.0234375,101.91306788)
\lineto(147.0234375,104.54099756)
\closepath
}
}
{
\newrgbcolor{curcolor}{0 0 0}
\pscustom[linestyle=none,fillstyle=solid,fillcolor=curcolor]
{
\newpath
\moveto(155.70703125,93.5634585)
\curveto(156.4453125,93.5634585)(157.05761719,93.87107569)(157.54394531,94.48631006)
\curveto(158.03613281,95.10740381)(158.28222656,96.03318506)(158.28222656,97.26365381)
\curveto(158.28222656,98.01365381)(158.17382812,98.65818506)(157.95703125,99.19724756)
\curveto(157.546875,100.23435694)(156.796875,100.75291163)(155.70703125,100.75291163)
\curveto(154.61132812,100.75291163)(153.86132812,100.20506006)(153.45703125,99.10935694)
\curveto(153.24023438,98.52341944)(153.13183594,97.77927881)(153.13183594,96.87693506)
\curveto(153.13183594,96.15037256)(153.24023438,95.5322085)(153.45703125,95.02244288)
\curveto(153.8671875,94.04978663)(154.6171875,93.5634585)(155.70703125,93.5634585)
\closepath
\moveto(151.61132812,101.86912256)
\lineto(153.14941406,101.86912256)
\lineto(153.14941406,100.62107569)
\curveto(153.46582031,101.04881006)(153.81152344,101.37986475)(154.18652344,101.61423975)
\curveto(154.71972656,101.96580225)(155.34667969,102.1415835)(156.06738281,102.1415835)
\curveto(157.13378906,102.1415835)(158.0390625,101.73142725)(158.78320312,100.91111475)
\curveto(159.52734375,100.09666163)(159.89941406,98.930646)(159.89941406,97.41306788)
\curveto(159.89941406,95.36228663)(159.36328125,93.89744288)(158.29101562,93.01853663)
\curveto(157.61132812,92.461896)(156.8203125,92.18357569)(155.91796875,92.18357569)
\curveto(155.20898438,92.18357569)(154.61425781,92.33884913)(154.13378906,92.649396)
\curveto(153.85253906,92.82517725)(153.5390625,93.12693506)(153.19335938,93.55466944)
\lineto(153.19335938,88.74705225)
\lineto(151.61132812,88.74705225)
\lineto(151.61132812,101.86912256)
\closepath
}
}
{
\newrgbcolor{curcolor}{0 0 0}
\pscustom[linewidth=1,linecolor=curcolor]
{
\newpath
\moveto(430.9595,110.50001246)
\lineto(550.9595,110.50001246)
\lineto(550.9595,80.50001246)
\lineto(430.9595,80.50001246)
\closepath
}
}
{
\newrgbcolor{curcolor}{0 0 0}
\pscustom[linestyle=none,fillstyle=solid,fillcolor=curcolor]
{
\newpath
\moveto(452.11965625,105.45506006)
\lineto(453.7016875,105.45506006)
\lineto(453.7016875,100.63865381)
\curveto(454.0766875,101.11326319)(454.41360156,101.44724756)(454.71242969,101.64060694)
\curveto(455.22219531,101.97459131)(455.8579375,102.1415835)(456.61965625,102.1415835)
\curveto(457.98489062,102.1415835)(458.91067187,101.66404444)(459.397,100.70896631)
\curveto(459.66067187,100.18748194)(459.79250781,99.46384913)(459.79250781,98.53806788)
\lineto(459.79250781,92.49998194)
\lineto(458.16653125,92.49998194)
\lineto(458.16653125,98.43259913)
\curveto(458.16653125,99.12400538)(458.07864062,99.63084131)(457.90285937,99.95310694)
\curveto(457.61575,100.46873194)(457.0766875,100.72654444)(456.28567187,100.72654444)
\curveto(455.62942187,100.72654444)(455.03469531,100.5009585)(454.50149219,100.04978663)
\curveto(453.96828906,99.59861475)(453.7016875,98.74607569)(453.7016875,97.49216944)
\lineto(453.7016875,92.49998194)
\lineto(452.11965625,92.49998194)
\lineto(452.11965625,105.45506006)
\closepath
}
}
{
\newrgbcolor{curcolor}{0 0 0}
\pscustom[linestyle=none,fillstyle=solid,fillcolor=curcolor]
{
\newpath
\moveto(462.45559375,104.54099756)
\lineto(464.05520312,104.54099756)
\lineto(464.05520312,101.91306788)
\lineto(465.55813281,101.91306788)
\lineto(465.55813281,100.62107569)
\lineto(464.05520312,100.62107569)
\lineto(464.05520312,94.477521)
\curveto(464.05520312,94.149396)(464.16653125,93.92966944)(464.3891875,93.81834131)
\curveto(464.51223437,93.75388819)(464.7173125,93.72166163)(465.00442187,93.72166163)
\lineto(465.25051562,93.72166163)
\curveto(465.33840625,93.727521)(465.44094531,93.73631006)(465.55813281,93.74802881)
\lineto(465.55813281,92.49998194)
\curveto(465.37649219,92.44724756)(465.1860625,92.40916163)(464.98684375,92.38572413)
\curveto(464.79348437,92.36228663)(464.58254687,92.35056788)(464.35403125,92.35056788)
\curveto(463.61575,92.35056788)(463.11477344,92.53806788)(462.85110156,92.91306788)
\curveto(462.58742969,93.29392725)(462.45559375,93.78611475)(462.45559375,94.38963038)
\lineto(462.45559375,100.62107569)
\lineto(461.18117969,100.62107569)
\lineto(461.18117969,101.91306788)
\lineto(462.45559375,101.91306788)
\lineto(462.45559375,104.54099756)
\closepath
}
}
{
\newrgbcolor{curcolor}{0 0 0}
\pscustom[linestyle=none,fillstyle=solid,fillcolor=curcolor]
{
\newpath
\moveto(467.16653125,101.91306788)
\lineto(468.73098437,101.91306788)
\lineto(468.73098437,100.57713038)
\curveto(469.10598437,101.040021)(469.44582812,101.37693506)(469.75051562,101.58787256)
\curveto(470.272,101.94529444)(470.86379687,102.12400538)(471.52590625,102.12400538)
\curveto(472.27590625,102.12400538)(472.87942187,101.93943506)(473.33645312,101.57029444)
\curveto(473.59426562,101.35935694)(473.82864062,101.04881006)(474.03957812,100.63865381)
\curveto(474.39114062,101.14256006)(474.80422656,101.51463038)(475.27883594,101.75486475)
\curveto(475.75344531,102.0009585)(476.28664844,102.12400538)(476.87844531,102.12400538)
\curveto(478.14407031,102.12400538)(479.00539844,101.66697413)(479.46242969,100.75291163)
\curveto(479.70852344,100.26072413)(479.83157031,99.59861475)(479.83157031,98.7665835)
\lineto(479.83157031,92.49998194)
\lineto(478.18801562,92.49998194)
\lineto(478.18801562,99.03904444)
\curveto(478.18801562,99.66599756)(478.0298125,100.09666163)(477.71340625,100.33103663)
\curveto(477.40285937,100.56541163)(477.022,100.68259913)(476.57082812,100.68259913)
\curveto(475.94973437,100.68259913)(475.41360156,100.47459131)(474.96242969,100.05857569)
\curveto(474.51711719,99.64256006)(474.29446094,98.94822413)(474.29446094,97.97556788)
\lineto(474.29446094,92.49998194)
\lineto(472.6860625,92.49998194)
\lineto(472.6860625,98.64353663)
\curveto(472.6860625,99.2822085)(472.60989062,99.74802881)(472.45754687,100.04099756)
\curveto(472.2173125,100.48045069)(471.76907031,100.70017725)(471.11282031,100.70017725)
\curveto(470.51516406,100.70017725)(469.97024219,100.46873194)(469.47805469,100.00584131)
\curveto(468.99172656,99.54295069)(468.7485625,98.70506006)(468.7485625,97.49216944)
\lineto(468.7485625,92.49998194)
\lineto(467.16653125,92.49998194)
\lineto(467.16653125,101.91306788)
\closepath
}
}
{
\newrgbcolor{curcolor}{0 0 0}
\pscustom[linestyle=none,fillstyle=solid,fillcolor=curcolor]
{
\newpath
\moveto(482.18703906,105.41111475)
\lineto(483.76907031,105.41111475)
\lineto(483.76907031,92.49998194)
\lineto(482.18703906,92.49998194)
\lineto(482.18703906,105.41111475)
\closepath
}
}
{
\newrgbcolor{curcolor}{0 0 0}
\pscustom[linewidth=1,linecolor=curcolor]
{
\newpath
\moveto(220.5,30.50001246)
\lineto(340.5,30.50001246)
\lineto(340.5,0.50001246)
\lineto(220.5,0.50001246)
\closepath
}
}
{
\newrgbcolor{curcolor}{0 0 0}
\pscustom[linestyle=none,fillstyle=solid,fillcolor=curcolor]
{
\newpath
\moveto(232.38964844,21.91306788)
\lineto(234.20019531,14.49509913)
\lineto(236.03710938,21.91306788)
\lineto(237.8125,21.91306788)
\lineto(239.65820312,14.53904444)
\lineto(241.58300781,21.91306788)
\lineto(243.16503906,21.91306788)
\lineto(240.43164062,12.49998194)
\lineto(238.78808594,12.49998194)
\lineto(236.87207031,19.78611475)
\lineto(235.01757812,12.49998194)
\lineto(233.37402344,12.49998194)
\lineto(230.65820312,21.91306788)
\lineto(232.38964844,21.91306788)
\closepath
}
}
{
\newrgbcolor{curcolor}{0 0 0}
\pscustom[linestyle=none,fillstyle=solid,fillcolor=curcolor]
{
\newpath
\moveto(248.58789062,22.12400538)
\curveto(249.25585938,22.12400538)(249.90332031,21.96580225)(250.53027344,21.649396)
\curveto(251.15722656,21.33884913)(251.63476562,20.93455225)(251.96289062,20.43650538)
\curveto(252.27929688,19.961896)(252.49023438,19.40818506)(252.59570312,18.77537256)
\curveto(252.68945312,18.34177881)(252.73632812,17.65037256)(252.73632812,16.70115381)
\lineto(245.83691406,16.70115381)
\curveto(245.86621094,15.74607569)(246.09179688,14.97849756)(246.51367188,14.39841944)
\curveto(246.93554688,13.82420069)(247.58886719,13.53709131)(248.47363281,13.53709131)
\curveto(249.29980469,13.53709131)(249.95898438,13.80955225)(250.45117188,14.35447413)
\curveto(250.73242188,14.67088038)(250.93164062,15.03709131)(251.04882812,15.45310694)
\lineto(252.60449219,15.45310694)
\curveto(252.56347656,15.10740381)(252.42578125,14.72068506)(252.19140625,14.29295069)
\curveto(251.96289062,13.87107569)(251.70507812,13.52537256)(251.41796875,13.25584131)
\curveto(250.9375,12.78709131)(250.34277344,12.47068506)(249.63378906,12.30662256)
\curveto(249.25292969,12.21287256)(248.82226562,12.16599756)(248.34179688,12.16599756)
\curveto(247.16992188,12.16599756)(246.17675781,12.59080225)(245.36230469,13.44041163)
\curveto(244.54785156,14.29588038)(244.140625,15.49119288)(244.140625,17.02634913)
\curveto(244.140625,18.53806788)(244.55078125,19.76560694)(245.37109375,20.70896631)
\curveto(246.19140625,21.65232569)(247.26367188,22.12400538)(248.58789062,22.12400538)
\closepath
\moveto(251.11035156,17.95798975)
\curveto(251.04589844,18.64353663)(250.89648438,19.19138819)(250.66210938,19.60154444)
\curveto(250.22851562,20.36326319)(249.50488281,20.74412256)(248.49121094,20.74412256)
\curveto(247.76464844,20.74412256)(247.15527344,20.48045069)(246.66308594,19.95310694)
\curveto(246.17089844,19.43162256)(245.91015625,18.7665835)(245.88085938,17.95798975)
\lineto(251.11035156,17.95798975)
\closepath
\moveto(248.43847656,22.1415835)
\lineto(248.43847656,22.1415835)
\closepath
}
}
{
\newrgbcolor{curcolor}{0 0 0}
\pscustom[linestyle=none,fillstyle=solid,fillcolor=curcolor]
{
\newpath
\moveto(254.56445312,25.45506006)
\lineto(256.10253906,25.45506006)
\lineto(256.10253906,20.77048975)
\curveto(256.44824219,21.22166163)(256.86132812,21.56443506)(257.34179688,21.79881006)
\curveto(257.82226562,22.03904444)(258.34375,22.15916163)(258.90625,22.15916163)
\curveto(260.078125,22.15916163)(261.02734375,21.75486475)(261.75390625,20.946271)
\curveto(262.48632812,20.14353663)(262.85253906,18.95701319)(262.85253906,17.38670069)
\curveto(262.85253906,15.89841944)(262.4921875,14.66209131)(261.77148438,13.67771631)
\curveto(261.05078125,12.69334131)(260.05175781,12.20115381)(258.77441406,12.20115381)
\curveto(258.05957031,12.20115381)(257.45605469,12.37400538)(256.96386719,12.7197085)
\curveto(256.67089844,12.92478663)(256.35742188,13.25291163)(256.0234375,13.7040835)
\lineto(256.0234375,12.49998194)
\lineto(254.56445312,12.49998194)
\lineto(254.56445312,25.45506006)
\closepath
\moveto(258.67773438,13.59861475)
\curveto(259.53320312,13.59861475)(260.171875,13.9384585)(260.59375,14.618146)
\curveto(261.02148438,15.2978335)(261.23535156,16.19431788)(261.23535156,17.30759913)
\curveto(261.23535156,18.2978335)(261.02148438,19.118146)(260.59375,19.76853663)
\curveto(260.171875,20.41892725)(259.54785156,20.74412256)(258.72167969,20.74412256)
\curveto(258.00097656,20.74412256)(257.36816406,20.477521)(256.82324219,19.94431788)
\curveto(256.28417969,19.41111475)(256.01464844,18.5322085)(256.01464844,17.30759913)
\curveto(256.01464844,16.4228335)(256.12597656,15.70506006)(256.34863281,15.15427881)
\curveto(256.76464844,14.11716944)(257.54101562,13.59861475)(258.67773438,13.59861475)
\closepath
}
}
{
\newrgbcolor{curcolor}{0 0 0}
\pscustom[linestyle=none,fillstyle=solid,fillcolor=curcolor]
{
\newpath
\moveto(263.546875,10.24998194)
\lineto(263.546875,11.13767725)
\lineto(273.55761719,11.13767725)
\lineto(273.55761719,10.24998194)
\lineto(263.546875,10.24998194)
\closepath
}
}
{
\newrgbcolor{curcolor}{0 0 0}
\pscustom[linestyle=none,fillstyle=solid,fillcolor=curcolor]
{
\newpath
\moveto(278.35644531,22.18552881)
\curveto(279.41699219,22.18552881)(280.27832031,21.92771631)(280.94042969,21.41209131)
\curveto(281.60839844,20.89646631)(282.00976562,20.008771)(282.14453125,18.74900538)
\lineto(280.60644531,18.74900538)
\curveto(280.51269531,19.3290835)(280.29882812,19.80955225)(279.96484375,20.19041163)
\curveto(279.63085938,20.57713038)(279.09472656,20.77048975)(278.35644531,20.77048975)
\curveto(277.34863281,20.77048975)(276.62792969,20.27830225)(276.19433594,19.29392725)
\curveto(275.91308594,18.65525538)(275.77246094,17.86716944)(275.77246094,16.92966944)
\curveto(275.77246094,15.98631006)(275.97167969,15.19236475)(276.37011719,14.5478335)
\curveto(276.76855469,13.90330225)(277.39550781,13.58103663)(278.25097656,13.58103663)
\curveto(278.90722656,13.58103663)(279.42578125,13.78025538)(279.80664062,14.17869288)
\curveto(280.19335938,14.58298975)(280.45996094,15.133771)(280.60644531,15.83103663)
\lineto(282.14453125,15.83103663)
\curveto(281.96875,14.58298975)(281.52929688,13.66892725)(280.82617188,13.08884913)
\curveto(280.12304688,12.51463038)(279.22363281,12.227521)(278.12792969,12.227521)
\curveto(276.89746094,12.227521)(275.91601562,12.67576319)(275.18359375,13.57224756)
\curveto(274.45117188,14.47459131)(274.08496094,15.59959131)(274.08496094,16.94724756)
\curveto(274.08496094,18.59959131)(274.48632812,19.88572413)(275.2890625,20.805646)
\curveto(276.09179688,21.72556788)(277.11425781,22.18552881)(278.35644531,22.18552881)
\closepath
\moveto(278.11035156,22.1415835)
\lineto(278.11035156,22.1415835)
\closepath
}
}
{
\newrgbcolor{curcolor}{0 0 0}
\pscustom[linestyle=none,fillstyle=solid,fillcolor=curcolor]
{
\newpath
\moveto(287.46191406,13.51951319)
\curveto(288.51074219,13.51951319)(289.22851562,13.915021)(289.61523438,14.70603663)
\curveto(290.0078125,15.50291163)(290.20410156,16.38767725)(290.20410156,17.3603335)
\curveto(290.20410156,18.23923975)(290.06347656,18.9540835)(289.78222656,19.50486475)
\curveto(289.33691406,20.37205225)(288.56933594,20.805646)(287.47949219,20.805646)
\curveto(286.51269531,20.805646)(285.80957031,20.43650538)(285.37011719,19.69822413)
\curveto(284.93066406,18.95994288)(284.7109375,18.06931788)(284.7109375,17.02634913)
\curveto(284.7109375,16.024396)(284.93066406,15.18943506)(285.37011719,14.52146631)
\curveto(285.80957031,13.85349756)(286.50683594,13.51951319)(287.46191406,13.51951319)
\closepath
\moveto(287.5234375,22.18552881)
\curveto(288.73632812,22.18552881)(289.76171875,21.78123194)(290.59960938,20.97263819)
\curveto(291.4375,20.16404444)(291.85644531,18.97459131)(291.85644531,17.40427881)
\curveto(291.85644531,15.88670069)(291.48730469,14.63279444)(290.74902344,13.64256006)
\curveto(290.01074219,12.65232569)(288.86523438,12.1572085)(287.3125,12.1572085)
\curveto(286.01757812,12.1572085)(284.98925781,12.59373194)(284.22753906,13.46677881)
\curveto(283.46582031,14.34568506)(283.08496094,15.52341944)(283.08496094,16.99998194)
\curveto(283.08496094,18.58201319)(283.48632812,19.84177881)(284.2890625,20.77927881)
\curveto(285.09179688,21.71677881)(286.16992188,22.18552881)(287.5234375,22.18552881)
\closepath
\moveto(287.47070312,22.1415835)
\lineto(287.47070312,22.1415835)
\closepath
}
}
{
\newrgbcolor{curcolor}{0 0 0}
\pscustom[linestyle=none,fillstyle=solid,fillcolor=curcolor]
{
\newpath
\moveto(293.74609375,21.91306788)
\lineto(295.31054688,21.91306788)
\lineto(295.31054688,20.57713038)
\curveto(295.68554688,21.040021)(296.02539062,21.37693506)(296.33007812,21.58787256)
\curveto(296.8515625,21.94529444)(297.44335938,22.12400538)(298.10546875,22.12400538)
\curveto(298.85546875,22.12400538)(299.45898438,21.93943506)(299.91601562,21.57029444)
\curveto(300.17382812,21.35935694)(300.40820312,21.04881006)(300.61914062,20.63865381)
\curveto(300.97070312,21.14256006)(301.38378906,21.51463038)(301.85839844,21.75486475)
\curveto(302.33300781,22.0009585)(302.86621094,22.12400538)(303.45800781,22.12400538)
\curveto(304.72363281,22.12400538)(305.58496094,21.66697413)(306.04199219,20.75291163)
\curveto(306.28808594,20.26072413)(306.41113281,19.59861475)(306.41113281,18.7665835)
\lineto(306.41113281,12.49998194)
\lineto(304.76757812,12.49998194)
\lineto(304.76757812,19.03904444)
\curveto(304.76757812,19.66599756)(304.609375,20.09666163)(304.29296875,20.33103663)
\curveto(303.98242188,20.56541163)(303.6015625,20.68259913)(303.15039062,20.68259913)
\curveto(302.52929688,20.68259913)(301.99316406,20.47459131)(301.54199219,20.05857569)
\curveto(301.09667969,19.64256006)(300.87402344,18.94822413)(300.87402344,17.97556788)
\lineto(300.87402344,12.49998194)
\lineto(299.265625,12.49998194)
\lineto(299.265625,18.64353663)
\curveto(299.265625,19.2822085)(299.18945312,19.74802881)(299.03710938,20.04099756)
\curveto(298.796875,20.48045069)(298.34863281,20.70017725)(297.69238281,20.70017725)
\curveto(297.09472656,20.70017725)(296.54980469,20.46873194)(296.05761719,20.00584131)
\curveto(295.57128906,19.54295069)(295.328125,18.70506006)(295.328125,17.49216944)
\lineto(295.328125,12.49998194)
\lineto(293.74609375,12.49998194)
\lineto(293.74609375,21.91306788)
\closepath
}
}
{
\newrgbcolor{curcolor}{0 0 0}
\pscustom[linestyle=none,fillstyle=solid,fillcolor=curcolor]
{
\newpath
\moveto(308.72265625,21.91306788)
\lineto(310.28710938,21.91306788)
\lineto(310.28710938,20.57713038)
\curveto(310.66210938,21.040021)(311.00195312,21.37693506)(311.30664062,21.58787256)
\curveto(311.828125,21.94529444)(312.41992188,22.12400538)(313.08203125,22.12400538)
\curveto(313.83203125,22.12400538)(314.43554688,21.93943506)(314.89257812,21.57029444)
\curveto(315.15039062,21.35935694)(315.38476562,21.04881006)(315.59570312,20.63865381)
\curveto(315.94726562,21.14256006)(316.36035156,21.51463038)(316.83496094,21.75486475)
\curveto(317.30957031,22.0009585)(317.84277344,22.12400538)(318.43457031,22.12400538)
\curveto(319.70019531,22.12400538)(320.56152344,21.66697413)(321.01855469,20.75291163)
\curveto(321.26464844,20.26072413)(321.38769531,19.59861475)(321.38769531,18.7665835)
\lineto(321.38769531,12.49998194)
\lineto(319.74414062,12.49998194)
\lineto(319.74414062,19.03904444)
\curveto(319.74414062,19.66599756)(319.5859375,20.09666163)(319.26953125,20.33103663)
\curveto(318.95898438,20.56541163)(318.578125,20.68259913)(318.12695312,20.68259913)
\curveto(317.50585938,20.68259913)(316.96972656,20.47459131)(316.51855469,20.05857569)
\curveto(316.07324219,19.64256006)(315.85058594,18.94822413)(315.85058594,17.97556788)
\lineto(315.85058594,12.49998194)
\lineto(314.2421875,12.49998194)
\lineto(314.2421875,18.64353663)
\curveto(314.2421875,19.2822085)(314.16601562,19.74802881)(314.01367188,20.04099756)
\curveto(313.7734375,20.48045069)(313.32519531,20.70017725)(312.66894531,20.70017725)
\curveto(312.07128906,20.70017725)(311.52636719,20.46873194)(311.03417969,20.00584131)
\curveto(310.54785156,19.54295069)(310.3046875,18.70506006)(310.3046875,17.49216944)
\lineto(310.3046875,12.49998194)
\lineto(308.72265625,12.49998194)
\lineto(308.72265625,21.91306788)
\closepath
}
}
{
\newrgbcolor{curcolor}{0 0 0}
\pscustom[linestyle=none,fillstyle=solid,fillcolor=curcolor]
{
\newpath
\moveto(327.43457031,13.51951319)
\curveto(328.48339844,13.51951319)(329.20117188,13.915021)(329.58789062,14.70603663)
\curveto(329.98046875,15.50291163)(330.17675781,16.38767725)(330.17675781,17.3603335)
\curveto(330.17675781,18.23923975)(330.03613281,18.9540835)(329.75488281,19.50486475)
\curveto(329.30957031,20.37205225)(328.54199219,20.805646)(327.45214844,20.805646)
\curveto(326.48535156,20.805646)(325.78222656,20.43650538)(325.34277344,19.69822413)
\curveto(324.90332031,18.95994288)(324.68359375,18.06931788)(324.68359375,17.02634913)
\curveto(324.68359375,16.024396)(324.90332031,15.18943506)(325.34277344,14.52146631)
\curveto(325.78222656,13.85349756)(326.47949219,13.51951319)(327.43457031,13.51951319)
\closepath
\moveto(327.49609375,22.18552881)
\curveto(328.70898438,22.18552881)(329.734375,21.78123194)(330.57226562,20.97263819)
\curveto(331.41015625,20.16404444)(331.82910156,18.97459131)(331.82910156,17.40427881)
\curveto(331.82910156,15.88670069)(331.45996094,14.63279444)(330.72167969,13.64256006)
\curveto(329.98339844,12.65232569)(328.83789062,12.1572085)(327.28515625,12.1572085)
\curveto(325.99023438,12.1572085)(324.96191406,12.59373194)(324.20019531,13.46677881)
\curveto(323.43847656,14.34568506)(323.05761719,15.52341944)(323.05761719,16.99998194)
\curveto(323.05761719,18.58201319)(323.45898438,19.84177881)(324.26171875,20.77927881)
\curveto(325.06445312,21.71677881)(326.14257812,22.18552881)(327.49609375,22.18552881)
\closepath
\moveto(327.44335938,22.1415835)
\lineto(327.44335938,22.1415835)
\closepath
}
}
{
\newrgbcolor{curcolor}{0 0 0}
\pscustom[linestyle=none,fillstyle=solid,fillcolor=curcolor]
{
\newpath
\moveto(333.71875,21.91306788)
\lineto(335.22167969,21.91306788)
\lineto(335.22167969,20.57713038)
\curveto(335.66699219,21.12791163)(336.13867188,21.52341944)(336.63671875,21.76365381)
\curveto(337.13476562,22.00388819)(337.68847656,22.12400538)(338.29785156,22.12400538)
\curveto(339.63378906,22.12400538)(340.53613281,21.65818506)(341.00488281,20.72654444)
\curveto(341.26269531,20.21677881)(341.39160156,19.48728663)(341.39160156,18.53806788)
\lineto(341.39160156,12.49998194)
\lineto(339.78320312,12.49998194)
\lineto(339.78320312,18.43259913)
\curveto(339.78320312,19.00681788)(339.69824219,19.4697085)(339.52832031,19.821271)
\curveto(339.24707031,20.4072085)(338.73730469,20.70017725)(337.99902344,20.70017725)
\curveto(337.62402344,20.70017725)(337.31640625,20.66209131)(337.07617188,20.58591944)
\curveto(336.64257812,20.45701319)(336.26171875,20.19920069)(335.93359375,19.81248194)
\curveto(335.66992188,19.50193506)(335.49707031,19.17966944)(335.41503906,18.84568506)
\curveto(335.33886719,18.51756006)(335.30078125,18.04588038)(335.30078125,17.430646)
\lineto(335.30078125,12.49998194)
\lineto(333.71875,12.49998194)
\lineto(333.71875,21.91306788)
\closepath
\moveto(337.43652344,22.1415835)
\lineto(337.43652344,22.1415835)
\closepath
}
}
{
\newrgbcolor{curcolor}{0 0 0}
\pscustom[linewidth=1,linecolor=curcolor]
{
\newpath
\moveto(280.5,322.49999507)
\lineto(280.5,272.49999507)
}
}
{
\newrgbcolor{curcolor}{0 0 0}
\pscustom[linestyle=none,fillstyle=solid,fillcolor=curcolor]
{
\newpath
\moveto(280.5,282.49999507)
\lineto(276.5,286.49999507)
\lineto(280.5,272.49999507)
\lineto(284.5,286.49999507)
\lineto(280.5,282.49999507)
\closepath
}
}
{
\newrgbcolor{curcolor}{0 0 0}
\pscustom[linewidth=1,linecolor=curcolor]
{
\newpath
\moveto(280.5,282.49999507)
\lineto(276.5,286.49999507)
\lineto(280.5,272.49999507)
\lineto(284.5,286.49999507)
\lineto(280.5,282.49999507)
\closepath
}
}
{
\newrgbcolor{curcolor}{0 0 0}
\pscustom[linewidth=1,linecolor=curcolor]
{
\newpath
\moveto(491.78541,175.78746236)
\lineto(491.78541,175.78746236)
}
}
{
\newrgbcolor{curcolor}{0 0 0}
\pscustom[linewidth=1,linecolor=curcolor]
{
\newpath
\moveto(491.48075015,160.78746236)
\lineto(490.5,112.49999507)
}
}
{
\newrgbcolor{curcolor}{0 0 0}
\pscustom[linestyle=none,fillstyle=solid,fillcolor=curcolor]
{
\newpath
\moveto(490.70306468,122.4979331)
\lineto(486.78511535,126.57833418)
\lineto(490.5,112.49999507)
\lineto(494.78346577,126.41588243)
\lineto(490.70306468,122.4979331)
\closepath
}
}
{
\newrgbcolor{curcolor}{0 0 0}
\pscustom[linewidth=1,linecolor=curcolor]
{
\newpath
\moveto(490.70306468,122.4979331)
\lineto(486.78511535,126.57833418)
\lineto(490.5,112.49999507)
\lineto(494.78346577,126.41588243)
\lineto(490.70306468,122.4979331)
\closepath
}
}
{
\newrgbcolor{curcolor}{0 0 0}
\pscustom[linewidth=1,linecolor=curcolor]
{
\newpath
\moveto(220.5,252.49999507)
\lineto(200.5,252.49999507)
\lineto(200.5,192.49999507)
}
}
{
\newrgbcolor{curcolor}{0 0 0}
\pscustom[linestyle=none,fillstyle=solid,fillcolor=curcolor]
{
\newpath
\moveto(200.5,202.49999507)
\lineto(196.5,206.49999507)
\lineto(200.5,192.49999507)
\lineto(204.5,206.49999507)
\lineto(200.5,202.49999507)
\closepath
}
}
{
\newrgbcolor{curcolor}{0 0 0}
\pscustom[linewidth=1,linecolor=curcolor]
{
\newpath
\moveto(200.5,202.49999507)
\lineto(196.5,206.49999507)
\lineto(200.5,192.49999507)
\lineto(204.5,206.49999507)
\lineto(200.5,202.49999507)
\closepath
}
}
{
\newrgbcolor{curcolor}{0 0 0}
\pscustom[linewidth=1,linecolor=curcolor]
{
\newpath
\moveto(220.5,254.49999507)
\lineto(60.5,254.49999507)
\lineto(60.5,194.49999507)
}
}
{
\newrgbcolor{curcolor}{0 0 0}
\pscustom[linestyle=none,fillstyle=solid,fillcolor=curcolor]
{
\newpath
\moveto(60.5,204.49999507)
\lineto(56.5,208.49999507)
\lineto(60.5,194.49999507)
\lineto(64.5,208.49999507)
\lineto(60.5,204.49999507)
\closepath
}
}
{
\newrgbcolor{curcolor}{0 0 0}
\pscustom[linewidth=1,linecolor=curcolor]
{
\newpath
\moveto(60.5,204.49999507)
\lineto(56.5,208.49999507)
\lineto(60.5,194.49999507)
\lineto(64.5,208.49999507)
\lineto(60.5,204.49999507)
\closepath
}
}
{
\newrgbcolor{curcolor}{0 0 0}
\pscustom[linewidth=1,linecolor=curcolor]
{
\newpath
\moveto(60.5,162.49999507)
\lineto(60.5,92.49999507)
\lineto(110.5,92.49999507)
}
}
{
\newrgbcolor{curcolor}{0 0 0}
\pscustom[linestyle=none,fillstyle=solid,fillcolor=curcolor]
{
\newpath
\moveto(100.5,92.49999507)
\lineto(96.5,88.49999507)
\lineto(110.5,92.49999507)
\lineto(96.5,96.49999507)
\lineto(100.5,92.49999507)
\closepath
}
}
{
\newrgbcolor{curcolor}{0 0 0}
\pscustom[linewidth=1,linecolor=curcolor]
{
\newpath
\moveto(100.5,92.49999507)
\lineto(96.5,88.49999507)
\lineto(110.5,92.49999507)
\lineto(96.5,96.49999507)
\lineto(100.5,92.49999507)
\closepath
}
}
{
\newrgbcolor{curcolor}{0 0 0}
\pscustom[linewidth=1,linecolor=curcolor]
{
\newpath
\moveto(339.64372,332.13358507)
\lineto(350.5,332.13358507)
\lineto(350.5,192.49999507)
}
}
{
\newrgbcolor{curcolor}{0 0 0}
\pscustom[linestyle=none,fillstyle=solid,fillcolor=curcolor]
{
\newpath
\moveto(350.5,202.49999507)
\lineto(346.5,206.49999507)
\lineto(350.5,192.49999507)
\lineto(354.5,206.49999507)
\lineto(350.5,202.49999507)
\closepath
}
}
{
\newrgbcolor{curcolor}{0 0 0}
\pscustom[linewidth=1,linecolor=curcolor]
{
\newpath
\moveto(350.5,202.49999507)
\lineto(346.5,206.49999507)
\lineto(350.5,192.49999507)
\lineto(354.5,206.49999507)
\lineto(350.5,202.49999507)
\closepath
}
}
{
\newrgbcolor{curcolor}{0 0 0}
\pscustom[linewidth=1,linecolor=curcolor]
{
\newpath
\moveto(339.64372,342.40888507)
\lineto(489.64372,342.40888507)
\lineto(490.5,192.49999507)
}
}
{
\newrgbcolor{curcolor}{0 0 0}
\pscustom[linestyle=none,fillstyle=solid,fillcolor=curcolor]
{
\newpath
\moveto(490.4428809,202.49983194)
\lineto(486.42009852,206.47691905)
\lineto(490.5,192.49999507)
\lineto(494.41996801,206.52261433)
\lineto(490.4428809,202.49983194)
\closepath
}
}
{
\newrgbcolor{curcolor}{0 0 0}
\pscustom[linewidth=1,linecolor=curcolor]
{
\newpath
\moveto(490.4428809,202.49983194)
\lineto(486.42009852,206.47691905)
\lineto(490.5,192.49999507)
\lineto(494.41996801,206.52261433)
\lineto(490.4428809,202.49983194)
\closepath
}
}
{
\newrgbcolor{curcolor}{0 0 0}
\pscustom[linewidth=1,linecolor=curcolor]
{
\newpath
\moveto(490.5,82.49999507)
\lineto(490.5,12.49999507)
\lineto(340.5,12.49999507)
}
}
{
\newrgbcolor{curcolor}{0 0 0}
\pscustom[linestyle=none,fillstyle=solid,fillcolor=curcolor]
{
\newpath
\moveto(350.5,12.49999507)
\lineto(354.5,16.49999507)
\lineto(340.5,12.49999507)
\lineto(354.5,8.49999507)
\lineto(350.5,12.49999507)
\closepath
}
}
{
\newrgbcolor{curcolor}{0 0 0}
\pscustom[linewidth=1,linecolor=curcolor]
{
\newpath
\moveto(350.5,12.49999507)
\lineto(354.5,16.49999507)
\lineto(340.5,12.49999507)
\lineto(354.5,8.49999507)
\lineto(350.5,12.49999507)
\closepath
}
}
{
\newrgbcolor{curcolor}{0 0 0}
\pscustom[linewidth=1,linecolor=curcolor]
{
\newpath
\moveto(60.5,162.49999507)
\lineto(60.5,12.49999507)
\lineto(220.5,12.49999507)
}
}
{
\newrgbcolor{curcolor}{0 0 0}
\pscustom[linestyle=none,fillstyle=solid,fillcolor=curcolor]
{
\newpath
\moveto(210.5,12.49999507)
\lineto(206.5,8.49999507)
\lineto(220.5,12.49999507)
\lineto(206.5,16.49999507)
\lineto(210.5,12.49999507)
\closepath
}
}
{
\newrgbcolor{curcolor}{0 0 0}
\pscustom[linewidth=1,linecolor=curcolor]
{
\newpath
\moveto(210.5,12.49999507)
\lineto(206.5,8.49999507)
\lineto(220.5,12.49999507)
\lineto(206.5,16.49999507)
\lineto(210.5,12.49999507)
\closepath
}
}
{
\newrgbcolor{curcolor}{0 0 0}
\pscustom[linewidth=1,linecolor=curcolor]
{
\newpath
\moveto(170.5,82.49999507)
\lineto(170.5,22.49999507)
\lineto(220.5,22.49999507)
}
}
{
\newrgbcolor{curcolor}{0 0 0}
\pscustom[linestyle=none,fillstyle=solid,fillcolor=curcolor]
{
\newpath
\moveto(210.5,22.49999507)
\lineto(206.5,18.49999507)
\lineto(220.5,22.49999507)
\lineto(206.5,26.49999507)
\lineto(210.5,22.49999507)
\closepath
}
}
{
\newrgbcolor{curcolor}{0 0 0}
\pscustom[linewidth=1,linecolor=curcolor]
{
\newpath
\moveto(210.5,22.49999507)
\lineto(206.5,18.49999507)
\lineto(220.5,22.49999507)
\lineto(206.5,26.49999507)
\lineto(210.5,22.49999507)
\closepath
}
}
{
\newrgbcolor{curcolor}{0 0 0}
\pscustom[linewidth=1,linecolor=curcolor]
{
\newpath
\moveto(490.5,162.49999507)
\lineto(490.5,132.49999507)
\lineto(410.5,132.49999507)
\lineto(411.35628,22.49999507)
\lineto(341.63158,23.35627507)
}
}
{
\newrgbcolor{curcolor}{0 0 0}
\pscustom[linestyle=none,fillstyle=solid,fillcolor=curcolor]
{
\newpath
\moveto(351.63082599,23.23347563)
\lineto(355.67964416,27.18405425)
\lineto(341.63158,23.35627507)
\lineto(355.5814046,19.18465746)
\lineto(351.63082599,23.23347563)
\closepath
}
}
{
\newrgbcolor{curcolor}{0 0 0}
\pscustom[linewidth=1,linecolor=curcolor]
{
\newpath
\moveto(351.63082599,23.23347563)
\lineto(355.67964416,27.18405425)
\lineto(341.63158,23.35627507)
\lineto(355.5814046,19.18465746)
\lineto(351.63082599,23.23347563)
\closepath
}
}
{
\newrgbcolor{curcolor}{0 0 0}
\pscustom[linewidth=1,linecolor=curcolor]
{
\newpath
\moveto(200.5,162.49999507)
\lineto(200.5,142.49999507)
\lineto(170.5,142.49999507)
\lineto(170.5,112.49999507)
}
}
{
\newrgbcolor{curcolor}{0 0 0}
\pscustom[linestyle=none,fillstyle=solid,fillcolor=curcolor]
{
\newpath
\moveto(170.5,122.49999507)
\lineto(166.5,126.49999507)
\lineto(170.5,112.49999507)
\lineto(174.5,126.49999507)
\lineto(170.5,122.49999507)
\closepath
}
}
{
\newrgbcolor{curcolor}{0 0 0}
\pscustom[linewidth=1,linecolor=curcolor]
{
\newpath
\moveto(170.5,122.49999507)
\lineto(166.5,126.49999507)
\lineto(170.5,112.49999507)
\lineto(174.5,126.49999507)
\lineto(170.5,122.49999507)
\closepath
}
}
{
\newrgbcolor{curcolor}{0 0 0}
\pscustom[linewidth=1,linecolor=curcolor]
{
\newpath
\moveto(280.5,242.49999507)
\lineto(280.5,32.49999507)
}
}
{
\newrgbcolor{curcolor}{0 0 0}
\pscustom[linestyle=none,fillstyle=solid,fillcolor=curcolor]
{
\newpath
\moveto(280.5,42.49999507)
\lineto(276.5,46.49999507)
\lineto(280.5,32.49999507)
\lineto(284.5,46.49999507)
\lineto(280.5,42.49999507)
\closepath
}
}
{
\newrgbcolor{curcolor}{0 0 0}
\pscustom[linewidth=1,linecolor=curcolor]
{
\newpath
\moveto(280.5,42.49999507)
\lineto(276.5,46.49999507)
\lineto(280.5,32.49999507)
\lineto(284.5,46.49999507)
\lineto(280.5,42.49999507)
\closepath
}
}
{
\newrgbcolor{curcolor}{0 0 0}
\pscustom[linewidth=1,linecolor=curcolor]
{
\newpath
\moveto(350.5,162.49999507)
\lineto(350.5,72.49999507)
\lineto(280.5,72.49999507)
}
}
{
\newrgbcolor{curcolor}{0 0 0}
\pscustom[linestyle=none,fillstyle=solid,fillcolor=curcolor]
{
\newpath
\moveto(290.5,72.49999507)
\lineto(294.5,76.49999507)
\lineto(280.5,72.49999507)
\lineto(294.5,68.49999507)
\lineto(290.5,72.49999507)
\closepath
}
}
{
\newrgbcolor{curcolor}{0 0 0}
\pscustom[linewidth=1,linecolor=curcolor]
{
\newpath
\moveto(290.5,72.49999507)
\lineto(294.5,76.49999507)
\lineto(280.5,72.49999507)
\lineto(294.5,68.49999507)
\lineto(290.5,72.49999507)
\closepath
}
}
{
\newrgbcolor{curcolor}{0 0 0}
\pscustom[linewidth=1,linecolor=curcolor]
{
\newpath
\moveto(200.5,162.49999507)
\lineto(200.5,142.49999507)
\lineto(280.5,142.49999507)
}
}
{
\newrgbcolor{curcolor}{0 0 0}
\pscustom[linestyle=none,fillstyle=solid,fillcolor=curcolor]
{
\newpath
\moveto(270.5,142.49999507)
\lineto(266.5,138.49999507)
\lineto(280.5,142.49999507)
\lineto(266.5,146.49999507)
\lineto(270.5,142.49999507)
\closepath
}
}
{
\newrgbcolor{curcolor}{0 0 0}
\pscustom[linewidth=1,linecolor=curcolor]
{
\newpath
\moveto(270.5,142.49999507)
\lineto(266.5,138.49999507)
\lineto(280.5,142.49999507)
\lineto(266.5,146.49999507)
\lineto(270.5,142.49999507)
\closepath
}
}
{
\newrgbcolor{curcolor}{0 0 0}
\pscustom[linewidth=1,linecolor=curcolor]
{
\newpath
\moveto(350.5,162.49999507)
\lineto(350.5,102.49999507)
\lineto(230.5,102.49999507)
}
}
{
\newrgbcolor{curcolor}{0 0 0}
\pscustom[linestyle=none,fillstyle=solid,fillcolor=curcolor]
{
\newpath
\moveto(240.5,102.49999507)
\lineto(244.5,106.49999507)
\lineto(230.5,102.49999507)
\lineto(244.5,98.49999507)
\lineto(240.5,102.49999507)
\closepath
}
}
{
\newrgbcolor{curcolor}{0 0 0}
\pscustom[linewidth=1,linecolor=curcolor]
{
\newpath
\moveto(240.5,102.49999507)
\lineto(244.5,106.49999507)
\lineto(230.5,102.49999507)
\lineto(244.5,98.49999507)
\lineto(240.5,102.49999507)
\closepath
}
}
{
\newrgbcolor{curcolor}{0 0 0}
\pscustom[linewidth=1,linecolor=curcolor]
{
\newpath
\moveto(490.5,162.49999507)
\lineto(490.5,142.49999507)
\lineto(360.5,142.49999507)
\lineto(360.5,92.49999507)
\lineto(233.34413,92.49999507)
}
}
{
\newrgbcolor{curcolor}{0 0 0}
\pscustom[linestyle=none,fillstyle=solid,fillcolor=curcolor]
{
\newpath
\moveto(243.34413,92.49999507)
\lineto(247.34413,96.49999507)
\lineto(233.34413,92.49999507)
\lineto(247.34413,88.49999507)
\lineto(243.34413,92.49999507)
\closepath
}
}
{
\newrgbcolor{curcolor}{0 0 0}
\pscustom[linewidth=1,linecolor=curcolor]
{
\newpath
\moveto(243.34413,92.49999507)
\lineto(247.34413,96.49999507)
\lineto(233.34413,92.49999507)
\lineto(247.34413,88.49999507)
\lineto(243.34413,92.49999507)
\closepath
}
}
{
\newrgbcolor{curcolor}{0 0 0}
\pscustom[linewidth=1,linecolor=curcolor]
{
\newpath
\moveto(300.5,322.49997507)
\lineto(300.5,302.49997507)
\lineto(420.5,302.49997507)
\lineto(420.5,62.49997507)
\lineto(300.5,62.49997507)
\lineto(300.5,32.49997507)
}
}
{
\newrgbcolor{curcolor}{0 0 0}
\pscustom[linestyle=none,fillstyle=solid,fillcolor=curcolor]
{
\newpath
\moveto(300.5,42.49997507)
\lineto(296.5,46.49997507)
\lineto(300.5,32.49997507)
\lineto(304.5,46.49997507)
\lineto(300.5,42.49997507)
\closepath
}
}
{
\newrgbcolor{curcolor}{0 0 0}
\pscustom[linewidth=1,linecolor=curcolor]
{
\newpath
\moveto(300.5,42.49997507)
\lineto(296.5,46.49997507)
\lineto(300.5,32.49997507)
\lineto(304.5,46.49997507)
\lineto(300.5,42.49997507)
\closepath
}
}
\end{pspicture}

  \caption{``With'' Structure of Packages.\\  ``With'' to packages outside of this repository (Ada and GNAT standard packages) are ignored.  ``With'' from spec and body treated the same.}
  \label{fig:With}
\end{figure}

\subsection{\texttt{web.adb}}
This is the main routine.  It initializes the internal items table and starts the web server.  If other initializations are needed, they could be added or called from here.

\subsection{\texttt{web\_common}}
This package contains a number of items that are available for use in other packages.  Items are put here instead of in \texttt{web\_server} in order to avoid circular dependancies.  The primary item from \texttt{web\_common} that is expected to be used by user software is the \texttt{params} hash map.  The package is instantiated as follows:
\begin{verbatim}
   package params is new Ada.Containers.Indefinite_Hashed_Maps
     (Element_Type => String,
      Key_Type => String,
      Hash => Ada.Strings.Hash_Case_Insensitive,
      Equivalent_Keys => Ada.Strings.Equal_Case_Insensitive);
\end{verbatim}
If your code needs access to parameters passed in a GET or POST request, this is how you will get access to those parameters.  There are some examples where this is used in the \texttt{internal} and \texttt{svg} packages.  Both parameters and headers use this hash map type.

Another hash map type is instantiated for internally generated items.  First, every internal item procedure has the same signature so an access type can be created as:
\begin{verbatim}
   --
   --  Define a type for the user procedures.  This is for a map used to map
   --  the internal procedures.  The parameters are:
   --    s - The stream to write output to.
   --    p - Any passed parameters from the HTTP request
   --    h - HTTP request headers.
   --
   type user_proc is access procedure (s : GNAT.Sockets.Stream_Access;
                                       p : params.Map;
                                       h : params.Map);
\end{verbatim}
Then the hash map package is instantiated as:
\begin{verbatim}
   --
   --  Instantiate a hashed map indexed by a string and containing procedure
   --  accesses.  Used as a table to identify which internal procedure to call
   --  for internal requests.
   --
   package proc_tables is new Ada.Containers.Indefinite_Hashed_Maps
     (Element_Type => user_proc,
      Key_Type => String,
      Hash => Ada.Strings.Hash_Case_Insensitive,
      Equivalent_Keys => Ada.Strings.Equal_Case_Insensitive);
\end{verbatim}
This map is populated and used in the \texttt{web\_server} package.

\subsection{\texttt{web\_server}}
This package contains the main web server software.  Since this package calls the user code, any attempt by user code to call items in this package will create a circular dependency.  Items that may be needed by user code should be in \texttt{web\_common}.  The item in this package that gets called with user code is the following procedure:
\begin{verbatim}
   --
   --  This is the web server.  In initializes the network interface and enters
   --  an infinite loop processing requests.  The passed paraemters are:
   --    internals   - A map of the names of internal item and procedure
   --    config_name - The name of the configuration file
   --    port        - The port to listen on
   --
   --  Note that this procedure never returns.  Eventually code should be added
   --       to shut the server down and exit.  It may also be turned into a task
   --       which may allow multiple servers to be run simultaneously.
   --
   procedure server(internals : bbs.web_common.proc_tables.Map;
                    config_name : String;
                    port : GNAT.Sockets.Port_Type);
\end{verbatim}
This procedure should be called after all other initialization is done.  It starts the web server and does not return.  At some point, it may get turned into a task so that other processing in the main task can proceed along with it.

\subsection{\texttt{html}}
This package contains routines to support the generation of HTML.  The visible routines are:
\begin{itemize}
  \item \verb|html_head(s : GNAT.Sockets.Stream_Access; title : String)|\\
  Generate a simple HTML heading with the specified title.
  \item \verb|html_head(s : GNAT.Sockets.Stream_Access; title : String; style : String)|\\
  Generate a simple HTML heading with the specified title and style sheet.
  \item \verb|html_end(s : GNAT.Sockets.Stream_Access; name : String)|\\
  Generate an ending for an HTML item using the file specified in \texttt{name}.
\end{itemize}

\subsection{\texttt{http}}
This package contains routines to support HTTP.  Currently GET and POST methods are supported with some minimal support for the OPTIONS method.  GET methods are supported for all items while POST methods are only supported for internally generated items.  If a file or an internally generated item with no parameters is being served, any supplied parameters are just ignored. Some of these routines are intended for use by user code and some are not.  The visible routines are:
\begin{itemize}
  \item \verb|ok(s : GNAT.Sockets.Stream_Access; txt: String)|\\
  Return code 200 OK for normal cases.
  \item \verb|not_found(s : GNAT.Sockets.Stream_Access; item: String)|\\
  Return code 404 NOT FOUND for when the requested item is not in the directory.
  \item \verb|internal_error(s : GNAT.Sockets.Stream_Access; file: String)|\\
  Return code 500 INTERNAL SERVER ERROR generally when unable to open the file for the specified item.  This means an item has been added to \texttt{config.txt} without adding the file to the system.
  \item \verb|not_implemented_int(s : GNAT.Sockets.Stream_Access; item: String)|\\
  Return code 501 NOT IMPLEMENTED for a request for an internally generated item that is not yet implemented.
  \item \begin{verbatim}read_headers(s : GNAT.Sockets.Stream_Access;
             method : out request_type;
             item : out Ada.Strings.Unbounded.Unbounded_String;
             params : in out web_common.params.Map)
            \end{verbatim}
   The \texttt{read\_headers} procedure handles both GET, POST, and OPTIONS request and returns any passed parameters.  Returned values will be the requested item in \texttt{item} and a dictionary containing the parameters in \texttt{params}.  If there are no parameters, the dictionary will be empty.  This routine is called by the web server to read the headers of the HTTP request.  By the time any user code is called, the headers have already been read.
\end{itemize}

\subsection{\texttt{internal}}
This package contains routines to generate HTML or XML for internally generated items.  The generated HTML or XML is written to the \texttt{Stream\_Access} that needs to be passed in.  The proper HTTP headers also need to be written.  The routines in this package are only referenced (if referenced at all) from the internal items map.  They do provide some examples of how to generate items.  The internal routines can have access to parameters passed in the GET or POST request as well as the HTTP headers.  If no parameters were passed, the parameters dictionary is empty.  The visible routines are:
\begin{itemize}
  \item \begin{verbatim}
procedure xml_count(s : GNAT.Sockets.Stream_Access;
                    h : web_common.params.Map;
                    p : web_common.params.Map);
\end{verbatim}
  Sends the count of transactions as an xml message.
  \item \begin{verbatim}
procedure html_show_config(s : GNAT.Sockets.Stream_Access;
                           h : web_common.params.Map;
                           p : web_common.params.Map);
\end{verbatim}
  Sends the configuration data as a HTML table.
  \item \begin{verbatim}
procedure target(s : GNAT.Sockets.Stream_Access;
                 h : web_common.params.Map;
                 p : web_common.params.Map);
\end{verbatim}
  Sends the parameters provided as a HTML table.
  \item \begin{verbatim}
procedure html_reload_config(s : GNAT.Sockets.Stream_Access;
                             h : web_common.params.Map;
                             p : web_common.params.Map);
\end{verbatim}
  Request that the configuration file be reloaded.  This can be useful during development and debugging.
\end{itemize}

\subsection{\texttt{svg}}
This package contains routines to generate SVG for internally generated items.  The generated SVG is written to the \texttt{Stream\_Access} that needs to be passed in.  The proper HTTP headers also need to be written.  The visible routines are:
\begin{itemize}
  \item \begin{verbatim}
procedure thermometer(s : GNAT.Sockets.Stream_Access;
                      h : web_common.params.Map;
                      p : web_common.params.Map);
\end{verbatim}
  Send SVG code to display a thermometer showing the value parameter.  This procedure handles getting and checking the parameters.  The following parameters are supported:
  \begin{itemize}
    \item \texttt{min} -- The minimum displayed value
    \item \texttt{max} -- The maximum displayed value
    \item \texttt{value} -- The value to display.
  \end{itemize}
  The value is clamped to be between \texttt{min} and \texttt{max}.  If any exceptions occur in parsing the parameters or if \texttt{min} is greater than \texttt{max}, a red ``X'' will be presented as the graphic to indicate an error condition.  Using the default configuration, this could be requested as \texttt{/Thermometer?min=0\&max=100\&value=50}
  \item \begin{verbatim}
procedure dial(s : GNAT.Sockets.Stream_Access;
               h : web_common.params.Map;
               p : web_common.params.Map);
\end{verbatim}
  Send SVG code to display a round dial with a pointer to the appropriate value. The following parameters are supported:
  \begin{itemize}
    \item \texttt{min} -- The minimum displayed value
    \item \texttt{max} -- The maximum displayed value
    \item \texttt{value} -- The value to display.
  \end{itemize}
  The value is clamped to be between \texttt{min} and \texttt{max}.  If any exceptions occur in parsing the parameters or if \texttt{min} is greater than \texttt{max}, a red ``X'' will be presented as the graphic to indicate an error condition.  Using the default configuration, this could be requested as \texttt{/Dial?min=0\&max=100\&value=50}
\end{itemize}

\subsection{\texttt{files}}
This package consists of a spec and body and is used to support the serving of binary and text files.  The visible routines are:
\begin{itemize}
  \item \begin{verbatim}send_binary_with_headers(s : GNAT.Sockets.Stream_Access;
                         mime : String; name : String);
            \end{verbatim}
  This procedure sends a binary file to the client with headers.  The file name is contained in the parameter \texttt{name} and the MIME type of the file is contained in the parameter \texttt{mime}.

  \item \begin{verbatim}send_text_with_headers(s : GNAT.Sockets.Stream_Access;
                       mime : String; name : String)
  \end{verbatim}
  This procedure sends a text file to the client with headers.  The file name is contained in the parameter \texttt{name} and the MIME type of the file is contained in the parameter \texttt{mime}.

  \item \begin{verbatim}send_text_without_headers(s : GNAT.Sockets.Stream_Access;
                          name : String)
  \end{verbatim}
  This procedure sends a text file to the client without headers.  The file name is contained in the parameter \texttt{name}.  No HTTP headers are sent.
\end{itemize}

\section{Configuration}
Generally before being used, the web server needs to be configured.  An example configuration is included and may be modified as needed.

\subsection{Configuration File}
The primary means of configuration is the \texttt{config.txt} file in the repository root.  This file is used to translate from the URL requested to the actual page served.  Both external files and internally generated responses are supported.

The format of the configuration file is fairly simple.  A line that has a pound sign, ``\#'' (octothorp) as the first character is a comment.  Non comment lines consist of three fields separated by a single space.
\begin{itemize}
  \item The first field is the requested URL minus the server specification.  Due to the nature of URLs, the first character will always be a slash, ``/''.  The web server uses a simple dictionary data structure for the URLs so the structure is technically flat.  However, any sort of hierarchical structure can be simulated.
  \item The second field identifies the item to be served.  It may be a file or it may be an arbitrary string to identify an internally generated item.  Files served are passed unchanged and both text and binary files are supported.
  \item The third field identifies the MIME type of the file being served.  If the third field is ``\texttt{internal}'', the item being served will be generated internally.  The value of the second field is used to select the proper internal routine to call.
\end{itemize}

\subsection{Modifying the Software}
To change the port used by the server, specify a different value when calling \texttt{web\_server.server}.

To change the number of tasks (threads) available for serving, change the \texttt{num\_handlers} constant in \texttt{web\_server.ads}.  The default value is 10, which should be adequate.  If memory is tight, it can be reduced. If higher performance is needed, this number can be increased.

\section{Modifications}
To add or change internally generated items, the code will need to be modified and recompiled.  In both cases, the first place to look in the software is the \texttt{build\_internal\_map} procedure in the \texttt{web\_server.adb} file.

\subsection{Modifications to Existing Items}
First, identify the routine to modify by looking at the \texttt{build\_internal\_map} procedure.  Then the appropriate files can be edited and the routine located.  Once the routine is located, any necessary modifications can be made.

\subsection{Adding New Items}
The very first thing is to decide what you can your item to do.  The existing software has examples of routines that generate HTML, XML, and SVG.  These can be used as models.  These are contained in the \texttt{internal} (for HTML and XML) and \texttt{svg} (for SVG) packages.  If you need to interface with other hardware or software, it will probably be best to add new packages for these interfaces.

\end{document}
